\section{Conclusions}

The \lhcb detector is a dedicated B-physics experiment at the \lhc designed to reconstruct rare \bquark quark decays.
The \lhcb detector was designed to be able to clearly separate primary and secondary vertices, reconstruct the tracks from $p-p$ collisions with high resolution 
and to clearly identify charged hadrons and muons using dedicated detectors. 
The sub-detectors comprising \lhcb and their excellent performance are presented, 
the result of which has enabled an integrated luminosity of 1.0\invfb to be collected throught 2011. % in Section~\ref{sec:sec:lhcb:det}.

Simulation of \bquark quark decays within the \lhcb detector is an important part of understanding the detector.
This allows the ability of the detector to select of rare \B decays to be evaluated and the effect such a selection has on the resulting data.
The organisation of the \lhcb software and the simulation is described along with the main differences between \lhcb data and simulation. 
Different methods to correct the IP resolution, the particle identification, the detector occupancy and the tracking efficiency are shown.
These methods allow the simulation to be corrected to accurately represent the \BdToKstmm candidates in the data.

The change in data-taking conditions between 2010 and 2011 required a redevelopment of the trigger used to select \bquark quark decays in \lhcb. 
Two options were explored, a cut-based algorithm and a multi-variate algorithm that combine 2,3 and 4 body combinations of track to form a potential \B candidate. 
These triggers use general features of the \B decays and basic reconstruction to selects $n$-body combinations of basic tracks.% using a succession of one-dimensional cuts.
The performance of the multi-variate trigger was shown of be better than the cut-based trigger.
This is due to the ability of the multi-variate trigger to use more correlations between the variables provided than the cut-based selection and this was the trigger selected for use in the 2011 data-taking period.

%In conclusion, a muon-specific topological trigger to select electroweak penguin and semi-leptonic decays was developed.
%A similar trigger which used a simplified BDT to select similar events was developed alongside. 
%By taking advantage of the correlations between different kinematic properties the 
%BDT trigger was shown to achieve the same level of signal efficiency with a drastic increase in background rejection.
%This \hlttwotopo trigger, including the \mutopo, was

