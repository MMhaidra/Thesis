% Frontmatter ----------------------------------------------
\chapter*{Declaration}

The work presented in this thesis was carried out between October 2009 and \monthname~\the\year. 
It is the result of my own studies, with the support of members of the Imperial College HEP group and 
the broader \lhcb collaboration. The work of others is explicitly referenced.

The following specific contributions were made by myself.
\begin{enumerate}
\item In Chapter~\ref{chap:lhcb} describing the \lhcb experiment, Section~\ref{sec:lhcb:trigdev} on the development of the trigger at \lhcb for electroweak penguin and semi-leptonic decays.
\item In Chapter~\ref{chap:kstmm}, describing the angular analysis of \BdToKstmm at \lhcb, 
Section~\ref{sec:kstmm:ac} detailing the acceptance correction and 
Section~\ref{sec:kstmm:sys}, giving the methods to assess the data-simulation corrections and the acceptance correction for possible sources of systematic uncertainty. 
\item In Chapter~\ref{chap:swave:theo}, the analysis of the effect of a \kpi S-wave in \BdToKstll.
\item In Chapter~\ref{chap:swave:meas}, the measurement of a \kpi S-wave in \BdToKstmm.
\end{enumerate}

This thesis has not been submitted for any other qualification. 
The measurements presented in Chapter~\ref{chap:kstmm} appear in Refs~\cite{LHCb-PAPER-2011-020} 
and~\cite{Aaij:2013iag}.
The results of the \kpi S-wave theoretical work appear in~\cite{Blake:2012mb}.
The measurements of the \kpi S-wave in \BdToKstmm have not been published but will be used 
in future analyses of electroweak penguin decays at \lhcb.

\begin{flushright} 
\emph{Alex Shires, \the\day~\monthname~\the\year}
\end{flushright}

\vspace*{\fill}



{\scriptsize The copyright of this thesis rests with the author and is made available under a 
Creative Commons Attribution Non-Commercial No Derivatives licence. 
Researchers are free to copy, distribute or transmit the thesis on the condition that they attribute it, 
that they do not use it for commercial purposes and that they do not alter, transform or build upon it.
 For any reuse or redistribution, researchers must make clear to others the license terms of this work
 (Imperial College London  PhD regulations 2012).}
 
\vspace*{\fill}

{\scriptsize All plots that are marked as coming from papers of the LHCb collaboration are available 
 under the CC-BY 3.0 license, \href{http://creativecommons.org/licenses/by/3.0/}{http://creativecommons.org/licenses/by/3.0/.}  
 }

\doublepage


\chapter*{Abstract}

The angular distribution of \BdToKstmm was studied using  
1.0\invfb of $pp$ collisions recorded at the \lhcb detector at the \lhc.
Angular observables are measured in five independent bins of the di-muon invariant mass squared, \qsq, 
and the theoretically interesting region from $1 < \qsq < 6 \gevgevcccc$.
The results are in good agreement with Standard Model predictions.

The contribution from a \kpi S-wave is included in the angular distribution of \BdToKpill.
The \kpi S-wave is shown to have an overall dilution effect on measurements of the \BdToKpill angular observables. 
For an S-wave contribution of 7\% between a \kpi invariant mass squared of $0.64<\psq<1.0\gevgevcccc$, there is a significant 
bias on the angular observables for dataset of over 500 events.
It is possible to remove this bias by incorporating the S-wave into the angular distribution
 and by fitting the \kpi mass spectrum. 

The fraction of the S-wave in \BdToKstmm, \FS, was analysed in seven bins of \qsq using 1.0\invfb of data from \lhcb.
The value of \FS in the region from $1 < \qsq < 6 \gevgevcccc$  and from $0.64<\psq<1.0\gevgevcccc$ 
was found to be $$\FS = 0.083^{+0.057}_{-0.048}(\mathrm{stat.})^{+0.018}_{-0.050} (\mathrm{syst.}).$$
In the regions where no S-wave is found, 95\% confidence limits are given.
These measurements show that the  \kpi S-wave will be a vital consideration for future measurements of \BdToKstll.

\doublepage


\chapter*{Acknowledgements}

I would like to express my gratitude to the Science and Technologies Facilities Council 
and to the European Organisation for Nuclear Research (\cern) for giving me the opportunity to pursue this PhD,
both in London and in Geneva.
Special thanks to Imperial College London for allowing me to do both my undergraduate and postgraduate studies at
such an illustrious institution. %I wouldn't have wanted it any other way.

I would not have been able to do any of this work without the hundreds of 
scientists and engineers who develop and run the \lhc and participate in \cern.
It was an honour to work within the \lhcb collaboration and be part of the the Rare Decays working group.
I have enjoyed working with the \BdToKstmm `task-force' of
 Ulrik Egede, Mitesh Patel, Thomas Blake, Nicola Serra, Michel de Cian and Chris Parkinson.
Special thanks to Tom for going beyond the call of duty to work with us.

Thank you to the Imperial College HEP group and especially to my supervisor Ulrik Egede, 
for his support over the years and for giving me the chance to work in HEP, 
both as an undergraduate and as a postgraduate.

Cheers to Bryn, Martyn, Nick, Rosie, Sam R, Sandra and Wren for life in Geneva, 
to Sam I for cake and sanity-restoring chat in London,
to the Orchestra Saint-Pierre Fusterie for a year of music in French
and to the Battersea Badgers for keeping the cricket alive.
Thanks to Carla for her love and support at the critical time. I can never thank you enough and look forward to being able to repay you in kind.
Thanks to Chris Parkinson for being an excellent colleague and friend; it just wouldn't have been the same without you.

\doublepage


% Contents lists ----------------------------------------------
\setcounter{tocdepth}{2} % for contents indexed down to subsection

\tableofcontents

