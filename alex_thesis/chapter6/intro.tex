\section{Introduction}
\label{sec:swave:intro}

With the acquisition of large data sets of \BdToKstll decays,
it is possible to study the validity of some of the assumptions that have been 
made to measure the angular observables in experiments to date. 
Nearly all theoretical papers to date use the 
assumption that for the \kpi system the natural width of
the $\Kstarz(892)$ can be ignored. 
This means there is no interference with other
\kpi resonances. Existing \BdToKstll analyses
consider \BdToKstll signal with \kpi candidates in a
narrow mass window around the $\Kstarz(892)$. 
However, in this region
there is evidence of a broad S-wave below the $\Kstarz(892)$
and higher mass states which decay strongly to \kpi, such as the S-wave $\Kstarzz(1430)$
and the D-wave $\Kstarzt(1430)$~\cite{PDG2012}. 
The best understanding of the low mass S-wave contribution comes from
the analysis of \kpi scattering at the LASS experiment~\cite{Aston:179353}. 

The interference of an S-wave in a predominantly P-wave system has
previously been used to disambiguate otherwise equivalent solutions
for the value of the \CP-violating phase in \Bz~\cite{Aubert:2004cp}
and \Bs~\cite{Aaij:2012eq} oscillations.
In the determination of $\varphi_s$ in the $\decay{\Bs}{\jpsi\phi}$ decay it was also shown
 that it is required to take the S-wave contribution into account 
~\cite{Xie:2009fs} and this has subsequently been done for the experimental 
measurements described in~\cite{CDF:2011af,Abazov:2011ry,LHCb-PAPER-2011-021}.
In Chapter~\ref{chap:kstmm}, the S-wave was included as a systematic error on the analysis.
Here, the \Kstarz is used for any neutral kaon state which decays to \kpi. 

In this Chapter, a generic \kpi S-wave contribution to \BdToKstll is included in the angular analysis.  
The explicit inclusion of a spin-0 S-wave and a spin-1 P-wave state in the 
\BdToKpill angular distribution is developed in Section~\ref{sec:swave:theo}.
The consequences of including a \kpi S-wave on the angular observables is shown in Section~\ref{sec:swave:theo:obs}.
% formalism set out in~\cite{Lu:2011jm} to explicitly include 
The impact of an S-wave contribution on the determination of the theoretical observables is evaluated in Section~\ref{sec:swave:testing}.
The minimum sample size in which ignoring a \kpi S-wave contribution contributs a significant bias to measurements of 
the angular observables is found in Section~\ref{sec:swave:ignore}
along with the minimum S-wave contribution needed to bias the angular observables.
Section~\ref{sec:swave:measurement} demonstrates how the S-wave contribution can be correctly taken into account and 
evaluates the effect of this on the angular observables.
