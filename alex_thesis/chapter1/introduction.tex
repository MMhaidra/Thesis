\chapter{Introduction}

Particle physics is the study of the fundamental constituents of matter and 
the behaviour of these components in terms of the forces between them.
The aim of particle physics is to determine the rules underlying the universe in terms of the world today 
and to provide an explanation for the history of the universe.
The development of modern day particle physics began around the turn of the $20^{\text{th}}$ century with 
the discovery of the internal structure in the atom and the exploration of the electromagnetic spectrum.
The experimental discovery of the fundamental particles comes from studying particle interactions and collisions, 
the products of which provide information about the underlying reaction that occurred.
%These reactions are produced using particle accelerators to provide consistent sources of photons, electrons and protons
% in a variety of different energy regions.
%This allows the creation of controlled conditions and enables the testing of theoretical predictions across wide energy scales.
At the time of writing, just after the turn of the the $21^{\text{st}}$ century,
most of the fundamental particles and forces discovered so far can be placed within the framework called the Standard Model.

The Standard Model (SM) of particle physics is a combination of quantum field theories for the electromagnetic force, the weak force and
 the strong force, unified by electroweak symmetry breaking and the Higgs mechanism with particle masses introduced through the Yukawa couplings.
The SM contains sixteen particles: twelve matter particles and four force particles~\cite{PDG2012}.
However, everyday behaviour is based on only four of these particles, the up and down quarks, the electron and the photon.
The great success of the SM of particle physics is that it both explains the behaviour of 
everyday interactions and also the behaviour of very high energy interactions, at least up to energies of order of one \tev.
However, there are two sources of problems with the SM, the first of which arises 
from the exact formalism used to describe it and the second is from cosmological 
observations. 
%These problems indicate that physics beyond the Standard Model is required to explain.
The problems from within the SM come from the number of arbitrary and finely tuned parameters in the model.
The exact mass of the Higgs, the exact nature of each of the electromagnetic, strong and weak couplings 
and the arbitrary mass hierarchy are all empirically determined.
Also, the incorporation of neutrino masses and the inclusion of quantum theories of gravity both introduce particles beyond the 
Standard Model.
The consistent predictive power of the SM despite these problems drives significant parts of research in particle physics.

Outside of particle physics, observations from cosmology and astrophysics present additional problems.
It has been observed that the universe is cooling and 
that, from measurements of the cosmic microwave background, 
the universe is flat, homogeneous and isotropic on large scales~\cite{Bennett:2012zja}.
These observations can be reconciled by the inflationary model, which proposes that 
the universe underwent a period of rapid expansion after formation, caused by the negative energy density
 of a yet undiscovered high energy scalar field~\cite{Dodelson:2003ft}.
Another problem is that that the observed dynamics of of matter within galaxies 
does not correspond to the dynamics expected from bodies of such masses~\cite{DMDISC}.
This can be reconciled by assuming either that there is some unknown matter which does not interact electromagnetically
 or that modifications are required to general relativity on large scales~\cite{D'Amico:2009df,Famaey:2011kh}.
The last problem from cosmology is the observation that the expansion of the universe is taking place at an increasing rate. 
There is even less basis for a solution to this than for the previous problems, but the 
negative energy density needed to explain this acceleration is larger than for the inflationary field~\cite{Martin:2012bt}.

The motivation for continued research into particle physics lies in both the unsatisfactory nature of arbitrary parameters 
and also in the reconciliation of the problems from observational cosmology.
Searching for physics beyond the SM takes place in two different ways. 
The direct approach aims to produce new particles from high energy collisions, whereas indirect searches look at
the influence that unknown particles can have on SM processes. 
%Direct production of particles can only explore energy ranges at the centre-of-mass of the collision.
The flavour sector of particle physics is concerned with the quarks and their interactions in the bound states they form.
The coupling of the quarks to the other forces and particles is an essential part of the SM and hence is 
an obvious place to search for deviations from expected behaviour that may indicate effects of new physics.
Indirect searches in the flavour sector look for the effects of massive particles which can have subtle higher-order effects on 
flavour physics observables.
Hadrons composed of \bquark quarks are a good test for these effects due to the difference in mass (and therefore 
the available energy) in transitions from \bquark quarks to lighter quarks.

Measurements of the \BdToKpimm system using data collected at \lhcb during 2011 are presented.
An overview of the formalism of the SM is given in Chapter~\ref{chap:formalism}. %and ways to incorporate effects beyond the Standard Model 
The theoretical description of the \BdToKpill decay is presented along with the status of contemporary measurements in  Chapter~\ref{chap:kstmm:theo}.
The \lhcb detector is described in Chapter~\ref{chap:lhcb} along with the work done to develop the trigger for \BdToKstmm decays.
The first and second angular analyses of \BdToKstmm at \lhcb %~\cite{LHCb-PAPER-2011-020,LHCb-CONF-2012-008} 
are detailed in Chapter~\ref{chap:kstmm}.
The effect of S-wave interference in the \BdToKpill system is presented in Chapter~\ref{chap:swave:theo} and the first measurements of the 
\kpi S-wave in \BdToKpimm using data from \lhcb are presented in Chapter~\ref{chap:swave:meas}.
