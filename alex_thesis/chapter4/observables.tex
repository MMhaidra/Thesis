\section{Angular observables}
\label{sec:kstmm:obs}

The contributions from the Wilson coefficients defined above
 can be measured by measuring the transversity amplitudes
 through an angular analysis of the \BdToKstll angular distribution.
Direct measurements of the transversity amplitudes are dependent on the values of the form factors
 which have a significant theoretical uncertainty.
To mitigate these uncertainties and allow measurements of the Wilson coefficients, angular observables
can be constructed from the transversity amplitudes that are independent of the two heavy-to-light form factors.
Many angular observables have been proposed for the decay 
\BdToKstll~\cite{Kruger:2005ep,AltmannshoferBall,Egede:2008uy,Bobeth:2010wg,Matias:2012xw}. 
These observables are combinations of the amplitudes which both minimise the uncertainty from the form factors 
and maximise the contribution from new physics models. 
So far the forward-backward asymmetry (\AFB), the fraction of the \Kstarz longitudinal polarisation (\FL ) 
and two combinations of the transverse amplitudes (\AT2 and \AIm ) have been 
measured~\cite{Aaltonen:2011cn,Aubert:2007hz,Aaltonen:2011ja,Aubert:2008ju,PhysRevLett.103.171801}.

\subsection{P-wave observables}

These observables are constructed from combinations of amplitudes and
are normalised to the sum of amplitudes for the P-wave state, given as 
\begin{align}
|A_{10}|^2 + |A_{1||}|^2 + |A_{1\bot}|^2 \, ,
\end{align}
where the generic combination of amplitudes $A_{Ji}A_{Ji}^{*}$ is defined for a spin $J$ and a polarisation (0,$||$,$\bot$) as
\begin{align}
\label{eq:ampsq}
A_{Ji} A_{Ji}^{*} &= A_{JiL} A_{JiL}^{*} + A_{JiR} A_{JiR}^{*} \, .
\end{align}
The forward-backward asymmetry of the dilepton system, \AFB, enters in the angular coefficient $I_6$ and
is defined in terms of the amplitudes as
\begin{align}
\label{eq:theoafb}
\AFB(\qsq) &= \frac{3}{2} \frac{  \Re(A_{1L||}A_{1L\bot}^{*}) - 
\Re(A_{1R||}A_{1R\bot}^{*})}{|A_{10}|^2 + |A_{1||}|^2 + |A_{1\bot}|^2 } \, .
\end{align}
In a similar way, \FL, \OS3 and \OS9 are defined as
\begin{equation}
\begin{split}
\FL(\qsq) &= \frac{  |A_{10}|^2 } {|A_{10}|^2 + |A_{1||}|^2 + |A_{1\bot}|^2} \\
\OS3 (\qsq) &= \frac{ |A_{1\bot}|^2 - |A_{1||}|^2 }{ |A_{10}|^2 + |A_{1||}|^2 + |A_{1\bot}|^2} \\
\OS9 (\qsq) &= \frac{  \Im(A_{1L||}A_{1L\bot}^{*}) - \Im(A_{1R||}A_{1R\bot}^{*}) }{|A_{10}|^2 + |A_{1||}|^2 + |A_{1\bot}|^2} 
\end{split}
\end{equation}
where \OS3 and \OS9 are related to the angular coefficients $I_3$ and $I_9$ respectively.
These theoretical observables are normalised to the sum of the P-wave amplitudes
and the factorisation of the amplitudes into matrix elements and the 
propagators removes the \psq dependence from these theoretical observables. 

In terms of the angular distribution, \AFB can also be expressed 
as the difference 
between the number of `forward-going' \mup and the number of 
`backward-going' \mup in the rest frame of the \Bd,
\begin{align}
\label{eq:expafb}
\left[ \int_0^1 - \int_{-1}^0 \right]  \text{d}\ctl
\frac{\text{d}\Gamma}{\text{d}\qsq \text{d}\ctl} / 
\frac{\text{d}\Gamma}{\text{d}\qsq} \, ,
\end{align}
which explains the name of the observable. 

\subsection{Transverse observables}
\label{sec:kstmm:reparam}

Angular observables which are normalised to only the transverse helicity amplitudes have
 been studied with the additional aim of reducing the theoretical uncertainties~\cite{Melikhov:1998cd,Kruger:2005ep}.
This is achieved by separating out the dependence on the  longitudinal amplitudes and their form factors from the calculation.
The main transverse observable is \AT2 which comes from the angular coefficient $I_3$,
\begin{align}
\AT2 (\qsq) &=  \frac{ |A_{1\bot}|^2 - |A_{1||}|^2  }{ |A_{1\bot}|^2 + |A_{1||}|^2 }  \, .
\end{align}
The observables associated with $I_6$ and $I_9$ can be similarly reparameterised~\cite{Becirevic:2011bp} 
 to give
\begin{equation}
\begin{split}
\ATRe(\qsq) &= \frac{ |A_{1\bot}|^2 - |A_{1||}|^2 }{ |A_{1||}|^2 + |A_{1\bot}|^2} \\
\ATIm(\qsq) &= \frac{  \Im(A_{1L||}A_{1L\bot}^{*}) \Im(A_{1R||}A_{1R\bot}^{*}) }{ |A_{1||}|^2 + |A_{1\bot}|^2} \, .
\end{split}
\end{equation}
These observables are correlated to $(1-\FL)$ when the the angular distribution is normalised to the sum of the P-wave amplitudes.

\subsection{\CP asymmetric angular observables}

Angular observables equivalent to \OS3 and \OS9 for the \CP antisymmetric angular distribution can by constructed
from the definition of $I_i - \bar{I}_i$.
Two \CP antisymmetric angular observables, \OA3 and \OA9 for the angular coefficients $I_3$ and $I_9$,
which can be compared to the \OS{i} angular observables 
\begin{align}
\OA3 &= \frac{1}{2} \frac{\left(I_3 - \bar{I}_3\right)}{ |A_{10}|^2 + |A_{1||}|^2 + |A_{1\bot}|^2} \, , \\
\OA9 &= \frac{1}{2} \frac{\left(I_9 - \bar{I}_9\right)}{ |A_{10}|^2 + |A_{1||}|^2 + |A_{1\bot}|^2}  \, .
\end{align}

\subsection{Relation to the Wilson coefficients}

Each of the observables is related to the Wilson coefficients through bi-linear combinations of the transversity amplitudes.
This means that there are terms proportional to the combinations $|\Ceff{9,10} \pm \Cpeff{9,10} |^2$ and $|\Ceff7 - \Cpeff7 |^2$.
Each of these terms is multiplied by the relevant \Kstarz form factors giving the \qsq dependence. 
This can be seen in the SM predictions for \AFB and \FL in Fig.~\ref{fig:otherexp}.





