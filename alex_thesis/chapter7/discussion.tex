\section{Conclusions}

The contribution from a \kpi S-wave to \BdToKstmm was analysed using 1.0\invfb of data collected at \lhcb at \sqs=7\tev in 2011. 
The central value of the S-wave fraction is non-zero in three bins of \qsq but all the measurements are compatible with no S-wave contribution.
An S-wave fraction of 0.08 has less than a 5\% effect on the measured values of \AFB and \FL in the range from $1<\qsq(\gevgevcccc)<6$ as described in Table 3 of Ref.~\cite{Aaij:2013iag}.
Although the values found here do not have a significant effect on the current analysis, any new measurements of \BdToKstmm must consider contamination from a \kpi S-wave.
This will add additional complications to the model of the angular distribution used to measure the angular observables in the form of additional parameters for both the signal and the background \kpi shape
as well as the interference between the S- and the P-wave.
This will therefore influence the precision that can be obtained on the angular observables and reduce the improvement gained from the increase in statistics with a larger dataset.
The dominant systematic effect in the current analysis comes from making the approximation that the phase space function is at the centre of the \qsq bin.
This can be improved by varying the phase space factor in the fit model based on the \qsq and \B mass of the \BdToKpimm candidate.

The accuracy of future measurements of the \kpi S-wave in \BdToKpimm can be improved by fitting \ctk to include the interference term between the S- and the P-wave.
This requires an improved acceptance correction, possible with either a larger simulation sample or by alternatively including the angular acceptance in the fit model.


