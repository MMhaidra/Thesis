\section{Introduction}

Measuring the contribution of a \kpi S-wave in \BdToKpimm is a requirement 
to understand biases in future measurements of the \BdToKstll angular distribution as shown in the previous chapter.
%The previous chapter shows that a \kpi S-wave could
%have a significant effect on the angular observables for datasets over 200 events.
%A measurement of the \kpi S-wave in \BdToKstmm will allow the inclusion of the S-wave 
%in the angular distribution, separating the P-wave angular observables.
% This will also improve the quality of fits by accommodating an asymmetry in \ctk.
There are no previous measurements of the S-wave in electroweak penguin decays. 
The closest related measurements of S-wave contributions in decays of \B mesons to
 a \kpimm final state are from \BdToJpsiKstar~\cite{Aubert:2004cp,LHCB-PAPER-2012-014}
which give a total \kpi S-wave fraction of 7\% in the mass window from 800 to 1000\mev.
However, the production  of the \kpi state in the electroweak penguins is different from \BdToJpsiKstar 
due to the different form factors.
%presence of right-handed \Kstarzz amplitudes and 

To measure the \kpi S-wave in \BdToKpimm, the formalism set out in Chapter~\ref{chap:swave:theo}
 was combined with the techniques developed in 
Chapter~\ref{chap:kstmm} and applied to the data collected at \lhcb in 2011.
In this chapter, the measurement proceeds as follows.
Firstly, the data and the simulation used to make the measurement are detailed in Section~\ref{sec:swave:meas:data}.
The selection of \BdToKpimm candidates in a wider range of \kpi masses is detailed in Section~\ref{sec:swave:meas:sel} and 
 the  acceptance correction for this wide \kpi mass range is described in Section~\ref{sec:swave:meas:ac}.
The model used to parametrise the distribution of \BdToKpimm candidates is detailed in 
Section~\ref{sec:swave:meas:angdist}.
%and consists of empirical descriptions for the \kpimm and \kpi mass distributions.
%and the theoretical \BdToKpimm angular distribution.
The method used to apply the model to the data and extract the S-wave fraction in the P-wave 
mass window is described in Section~\ref{sec:swave:meas:fitmethod}.
Sources of systematic uncertainty and possible biases to the measurement 
are given in Sec.~\ref{sec:swave:meas:sys} and 
the results of the measurement of the \kpi S-wave contribution to \BdToKpimm 
are presented in Section~\ref{sec:swave:meas:res}.

