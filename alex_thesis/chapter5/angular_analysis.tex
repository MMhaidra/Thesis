\section{Angular analysis}
\label{sec:kstmm:pdf}

Each of the angular analyses were performed by simultaneously fitting a \PDF for the mass and the angular distribution 
to the data. % the extended maximum likelihood fit method.
%The \PDF used is a combination of a model for the the \Bd mass spectrum and the angular \PDF.
The simultaneous fit to the \Bd mass spectrum and to the angles ensures that the
 maximum information available is used to reduce the error on all of the angular observables.
It also ensures that the correlations are propagated correctly between the angular observables.
The total \PDF ($F$) is a combination of a model for the signal ($S$) and background ($B$) ,
each containing component \PDFs to describe the mass distribution and the angular distribution,
\begin{align}
F(\mB, \ctl, \ctk, \phi) = & f_{sig} \left( S_{i}(\mB) \times S_{i}(\ctl,\ctk,\phi) \right) \nonumber \\ 
&+ ( 1 - f_{sig} ) \left( B_{i}(\mB) \times B_{i}(\ctl,\ctk,\phi) \right) \, .
\end{align}
where $i$ indicates the model used for the first or second analysis.
The different components of the total \PDF are described in detail below.

The dataset is divided into seven bins of \qsq.
 There are six separate bins in the full \qsq range, detailed in Table~\ref{tbl:qsqbins}.
\begin{table}
\centering
\caption[ The \qsq binning scheme used in both angular analyses. ]
{ The \qsq binning scheme used in both angular analyses. 
The binning is analogous to the binning used in Ref~\cite{PhysRevLett.103.171801} including
 the  \qsq region of 1 to 6 \gevgevcccc .~\label{tbl:qsqbins} }
\begin{tabular}{|c|c|}
\hline
lower limit (\gevgevcccc)  & upper limit (\gevgevcccc) \\
\hline
0.1 & 2 \\
2 & 4.3 \\
4.3 & 8.68 \\
10.09 & 12.9 \\
14.18 & 16 \\
16 & 19 \\
\hline
1.0 & 6.0 \\
\hline
\end{tabular}
\end{table}
The binning is analogous to the binning used in Ref~\cite{PhysRevLett.103.171801} along with the
 region from $1 < \qsq < 6 \gevgevcccc $, which is a theoretically clean region where the observables are easily calculable.
The binning is chosen such that there is a bin below and above the point where \AFB is
 predicted to change sign in the Standard Model, and also with boundaries to avoid the \ccbar resonances.

\subsection{Mass model}
\label{sec:kstmm:massmodel}

Two different mass models were used to parametrise the \Bd signal invariant mass distribution.
The signal invariant mass model used to parametrise the \Bd invariant mass distribution in the first angular analysis was a double Gaussian function,
\begin{align}
\label{eq:totpdf}
S_{(1)}\left( \mkpimm ; \sigma_1, \sigma_2, \alpha, n \right) = & \, f \times G\left(  \mkpimm  ; m_B , \sigma_1 \right) \nonumber\\
& + \left(1-f\right) \times G\left(  \mkpimm ; m_B , \sigma_2\right) \, ,
\end{align}
where $f$ is the fraction of signal between each component and $\sigma_{1,2}$ are the different widths of each Gaussian component.
The signal mass model for the second angular analysis was an empirical model consisting of two Crystal Ball functions.
The Crystal Ball was a function developed to model the radiative 
tail from the \bbbar resonances~\cite{Skwarnicki:1986xj}.
It consists of a Gaussian distribution with an 
exponential tail and is expressed for a given mass ( \mkpimm ) as
\begin{equation}
\CB \left(  \mkpimm ; \mB , \sigma, \alpha, n \right) = N
\begin{cases}
  \exp{\left(\frac{-(\mkpimm - \mB )^2 }{2\sigma^2}\right)} 
 & \text{if} \,\, \mkpimm >  \alpha  \\
 \frac{\left(\frac{n}{\alpha^2}\right)^n}{\frac{\left(\mkpimm-\mB \right)}{\sigma} + \frac{n}{\alpha} - \alpha }
 & \text{if} \,\, \mkpimm\leq  \alpha
\end{cases}
\end{equation}
where $N$ is the signal normalisation, \mB is the nominal B mass, $\sigma$ is the Gaussian width and $n$ and $\alpha$ are the tail parameters.
Here the Crystal Ball function is used as an empirical formula to describe tails in the \Bd mass spectrum from resolution effects.
The parameters for the \Bd mass signal shape are assumed to be equivalent for both Crystal Ball functions except for the widths,
\begin{align}
S_{(2)}\left( \mkpimm ; \sigma_1, \sigma_2, \alpha, n \right) = & 
\, f \, \CB\left(  \mkpimm  ; m_B , \sigma_1, \alpha, n \right) \nonumber\\
& + \left(1-f\right) \times \CB\left(  \mkpimm ; m_B , \sigma_2, a, n \right) \ .
\end{align}

The shape of the signal mass model for both analyses is taken from fits to the \BdToJpsiKstar invariant mass spectrum. 
Due to the high statistics of \BdToJpsiKstar in the data, it is necessary to include an 
 additional contribution from the suppressed \BsToJpsiKstar mode.
The decay \BsToJpsiKstar is suppressed by a factor of \fd\Vtd/\Vts\fs compared to \BdToJpsiKstar.
The model used for the \BsToJpsiKstar is identical to the model for the \BdToJpsiKstar except for the central 
mass value and shares all of it's parameters with the \BdToJpsiKstar model. 
The only remaining free parameter is the relative normalisation between the two contributions.
There is a relative factor
\begin{align}
\frac{n_{\Bd}}{n_{\Bs}} =  0.007\pm0.002 \, ,
\end{align}
which is applied as a Gaussian constraint on the overall size of the \BsToJpsiKstar contribution.

The model for the background contribution to the \mkpimm spectrum for both analyses is the same. This is an exponential function,
\begin{align}
B_{(1,2)}\left(\mkpimm;\lambda\right) = N_B \exp{\left(-\lambda\mkpimm\right)} \, ,
\end{align}
where $\lambda$ is the decay constant for the exponential and $N_B$ is the normalisation of the background \PDF.

\subsection{Angular model}
\label{sec:kstmm:angularmodel}

The signal angular model for each of the analyses is a simplification of the full angular distribution for \BdToKstll 
as described in Sec.~\ref{sec:fullangdist}.
The angular distribution is integrated over one bin of \psq and integrated over each of the six bins of \qsq.
The signal model used in the 0.38\invfb angular analysis to measure \AFB and \FL is 
%a function of the angles \ctl and \ctk and integrated over $\phi$.% containing \AFB and \FL. 
%Integrating Eq.~\ref{eq:theo5d} over \psq and \varphi gives
\begin{align}
S_{(1)}(\ctl,\ctk) = & \frac{9}{16}   \Bigg( 2 \FL \ctksq ( 1 - \ctlsq )   \nonumber\\ 
& + \frac{1}{2} ( 1 - \FL ) ( 1 - \ctksq ) ( 1 + \ctlsq )  \nonumber\\
& +  \frac{4}{3} \AFB ( 1 - \ctksq ) \ctl   \Bigg) . 
\end{align}
The angular distribution for the 1.0\invfb angular analysis was extended to include angular observables dependent on \varphi. 
The distribution was simplified using the transformation described in Sec.~\ref{sec:fullangdist} and~\cite{Ksteepubnote}.
The analysis uses two parameterisations  of the angular distribution.
%, the first of which contains the regular angular observables from~\cite{AltmannshoferBall,Kruger:2005ep}.
The angular distribution for \AFB, \FL, \OS3 and \OS9 is given by
\begin{align}
\label{eq:theo4doriginal}
S_{(2a)}(\ctl,\ctk,\phiprime) =   \frac{9}{16\pi}  & \Bigg(  2 \FL \ctksq ( 1 - \ctlsq )  \nonumber \\ 
&  + \frac{1}{2} ( 1 - \FL ) ( 1 - \ctksq ) ( 1 + \ctlsq )  \nonumber \\
&  + \OS3 ( 1 - \ctksq ) (  1 - \ctlsq ) \cos2\phiprime  \nonumber \\
&  +  \frac{4}{3} \AFB ( 1 - \ctksq ) \ctl  \nonumber \\ 
&  + \OS9 ( 1 - \ctksq ) ( 1 - \ctlsq) \sin2\phiprime  \Bigg) . 
\end{align}
The re-parametrised angular distribution contains the transverse angular observables (\ATRe, \AT2, \ATIm) as described in Sec.~\ref{sec:kstmm:obs},
\begin{align}
\label{eq:theo4dreparam}
S_{(2b)}(\ctl,\ctk,\phiprime) =  \frac{9}{16\pi}  & \Bigg(  2 \FL \ctksq ( 1 - \ctlsq )  \nonumber \\ 
&  + \frac{1}{2} ( 1 - \FL ) ( 1 - \ctksq ) ( 1 + \ctlsq )  \nonumber \\
&  + \frac{1}{2} ( 1 - \FL ) \AT2 ( 1 - \ctksq ) (  1 - \ctlsq ) \cos2\phiprime  \nonumber \\
& +  \frac{4}{3}  ( 1 - \FL ) \ATRe ( 1 - \ctksq ) \ctl  \nonumber \\ 
& +  ( 1 - \FL ) \ATIm ( 1 - \ctksq ) ( 1 - \ctlsq) \sin2\phiprime  \Bigg) . 
\end{align}
The angular distribution used to measure \OA9 uses the CP anti-symmetric definition of \varphi where the sign changes
for \Bd and \Bdb decays as given in Sec~\ref{sec:kstmm:obs}.
The signal angular distribution is a function of \phiprimecp,
\begin{align}
\label{eq:theo4dphi}
S_{(2c)}(\ctl,\ctk,\phiprimecp) =   \frac{9}{16\pi}  & \Bigg(  2 \FL \ctksq ( 1 - \ctlsq )  \nonumber \\ 
&  + \frac{1}{2} ( 1 - \FL ) ( 1 - \ctksq ) ( 1 + \ctlsq )  \nonumber \\
&  + \OA3 ( 1 - \ctksq ) (  1 - \ctlsq ) \cos2\phiprimecp  \nonumber \\
&  +  \frac{4}{3} \AFB ( 1 - \ctksq ) \ctl  \nonumber \\ 
&  + \OA9 ( 1 - \ctksq ) ( 1 - \ctlsq) \sin2\phiprimecp  \Bigg) . 
\end{align}

The model for the background in each of the angles is equivalent for both angular analyses.
The background \PDF is an $n^{\text{th}}$ order Chebychev polynomial of the first kind for each angle,
\begin{align}
T_{n}(x) = \cos\left(n\arccos(x)\right) \, .
\end{align}
The total background angular \PDF is factorised into each of the angles,
\begin{align}
B(\mkpimm) = P`^{bkg}_{n}(\ctl, \ctk, \phiprime) = P_{n}^{L}(\ctl) \times P_{n}^{K}(\ctk) \times P_{n}^{P}\phiprime) \, .
\end{align}
The assumption that the background angular distribution factorises was tested using the point-to-point dissimilarity test~\cite{Williams:2010vh}.
The probability of the test statistic having a value less than the test statistic of the data was 25\%.
This value is entirely compatible with the assumption that the background factorises into the three angles.

\subsection{Result extraction}
\label{sec:kstmm:resextraction}

The signal \PDF is fitted to the data by performing an unbinned maximum-log-likelihood fit to the data,
minimising
\begin{align}
 - \log \mathcal{L} = \sum_{i}^{N}  \omega_i F(\mkpimm^i, \ctl^i, \ctk^i, \phi^i,  \vec{p} , \vec{O} ) ,
\end{align}
where $F$ is the total \PDF described in Eq.~\ref{eq:totpdf}. % which in terms of the \kpimm invariant mass and angles of each of the candidates.
The set of parameters for the signal and background mass models are $\vec{p}$, while $\vec{O}$ is the set of angular observables.
Each of the data candidates is weighted for acceptance as described in Section~\ref{sec:kstmm:ac}.
These weights distort the shape of the likelihood such that the errors extracted from the standard NLL minimisation are not 
guaranteed to be the true errors.
In each angular analysis, two different techniques were used to extract the likelihood minima and a better estimate of the error from the likelihood function.
In the 0.38\invfb analysis, the profile likelihood was calculated and the error determined from the two-dimensional 68\% confidence interval in both \AFB and \FL.
For the 1.0\invfb analysis, the errors were extracted in a Frequentist manner using the Feldman-Cousins (FC) technique~\cite{PhysRevD.57.3873}.

The FC technique maps out the likelihood for an observable, allowing the size of the confidence intervals for a given observable to be calculated. 
For an observable of interest in a given set of parameters, the ratio between the likelihood calculated with all parameters free $(\mathcal{L}_0)$ and the likelihood calculated with the observable fixed is calculated $(\mathcal{L}_1)$.
The ratio between these likelihood ($R_{data}$) is obtained for the result obtained from data, and for a large ensemble of toy datasets ($R_i$).
The fraction of $R_i < R_{data}$ ($f_R$) is proportional to the probability of the data result being the most optimum solution in the phase space of the parameters.
This fraction is calculated for a range of values for the observable and the 68\% confidence limits on the observable are calculated from the points where the $f_R<0.68$.

The results of the angular fits along with the calculated confidence limits are shown in Section~\ref{sec:kstmm:res}.

