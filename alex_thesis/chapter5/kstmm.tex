\chapter{The angular analysis of \BdToKstmm}
\label{chap:kstmm}

\emph{This chapter contains the work of the \lhcb collaboration. The author contributed to Section~\ref{sec:kstmm:ac}
and Section~\ref{sec:kstmm:sys}. The results in this chapter were published in Refs~\cite{LHCb-PAPER-2011-020} 
and~\cite{Aaij:2013iag}. The first analysis is presented as the author contributed to the acceptance correction 
and the results were the first measurements of electroweak penguins at \lhcb. }


\section{Introduction}

The angular analyses presented in this chapter are
the first and second angular analyses of \BdToKstmm performed at \lhcb. 
The first angular analysis concentrates on measuring the values of 
\AFB, \FL and the differential branching fraction in seven bins of dimuon mass. 
This was performed on the first 0.38\invfb of  data recorded at \lhcb in 2011. 
The second angular analysis is an extension of the first to encompass the 
angular observables dependent on \varphi. This allowed the measurement of \OS3, \OS9 and \OA9
 as well as the transverse angular observables, \ATRe, \ATIm and \AT2 .
This analysis uses the complete dataset of 1.0\invfb recorded in 2011 at \lhcb. 

Both analyses followed three main steps to obtain the values of the angular observables 
and the differential branching fraction in bins of \qsq.
A cut-based selection and a multi-variate discriminant are used to select signal \BdToKstmm candidates from the data.
Subsequently, the selected \BdToKstmm candidates were corrected for the acceptance effect introduced from their reconstruction and selection.
Finally, the weighted data was simultaneously fitted with a \PDF describing the
 \Bd invariant mass distribution and the angular distribution to determine the results.

This chapter follows the structure of the analysis in a similar manner.
The data for each analysis and the simulation used in this analysis are described in 
Sec.~\ref{sec:kstmm:data}. 
The selection of the \kpimm candidates for both analyses is described in Sec.~\ref{sec:kstmm:sel}. 
The two different methods used for the acceptance correction are detailed in Sec.~\ref{sec:kstmm:ac}.
The \PDF used to determine the angular observables for each analysis and the method of determining the errors
 is described in Sec.~\ref{sec:kstmm:pdf}. 
The estimates of the contribution from systematic effects are detailed in Sec~\ref{sec:kstmm:sys}. 
The results for the angular observables and for the differential branching fraction 
from both angular analyses are presented in Sec.~\ref{sec:kstmm:res}.

\section{Data samples}
\label{sec:kstmm:data}

This section describes the data and simulation samples used in the angular analysis of \BdToKstmm.
The second set of data is a superset of the first but processed with a later version of the reconstruction and event selection software.
There are two distinct versions of simulated events, one representing the data-taking conditions
and detector knowledge at the end of 2010 (MC10) and the second representing the equivalent conditions for the 2011 data-taking period (MC11).
The MC11 samples were only used in the second angular analysis.

\subsection{Data}

\subsubsection{Sample 1 - 0.38\invfb }

The  dataset used in the first analysis of \BdToKstmm at \lhcb was collected between March and June 2011.
The data was taken at a centre-of-mass energy of $\sqs=7\tev$ using both polarities of the \lhcb magnet.
The data sample corresponds to an integrated luminosity of 0.38\invpb.
The vast majority of data was taken in the trigger configuration using the multi-variate topological trigger % labelled \verb=0x006d0032=,
 with smaller samples being taken in almost identical conditions throughout the year.
The data are reconstructed with reconstruction version \verb=Reco10=, as described in Section~\ref{sec:lhcb:soft},
 and stripped with version \verb=Stripping13b=, described in detail below.

\subsubsection{Sample 2 - 1.0\invfb }

The dataset used in the second angular analysis of \BdToKstmm at \lhcb was the full dataset from the 2011 run of the \lhc.
This corresponds to an integrated luminosity of 1.0\invfb of data at a centre-of-mass energy of $\sqs=7\tev$.
The trigger configuration was consistent throughout 2011 for the trigger lines used to select events in the angular analysis.
The reconstruction and the event selection are consistent for the whole dataset.
The particular versions of the reconstruction and event selection software used are \verb=Reco12= and \verb=Stripping17= respectively.


\subsection{Simulation}
\label{sec:kstmm:data:mc}

The samples of simulation used in the angular analysis were generated as outlined in Sec.~\ref{sec:lhcb:soft}.
The generation and reconstruction conditions of each sample are described in detail below.
To ensure that the correct efficiency is calculated from the simulation, the properties 
of the simulation are compared with large data control samples.
In order to update the simulations to agree with the best knowledge of the detector in 2011,
the set of corrections derived in Section~\ref{sec:kstmm:data:mccorr} was checked using \BdToJpsiKstar data and simulation.

\subsubsection{MC10}


The samples of MC10 that were used for both angular analyses were 
simulated to be a close approximation of the data-taking conditions 
in 2010. % by using the latest versions of the trigger, reconstruction, and stripping software.
In order to use this simulation with the 2011 data, an updated version of the trigger and
event selection software was re-run over the simulated events.
The sample of simulated events generated in the MC10 configuration was of \BdToKstmm events.
This sample was generated using a decay model such that the events are flat in phase space and therefore 
have a uniform distribution in \ctl, \ctk and $\phi$. 
The distribution of phase space events in \qsq decreases as 
the size of the phase space available for the decay at higher \qsq values gets smaller.
The generator level distribution for \qsq is shown in Fig.~\ref{fig:phspqsq}.
\begin{figure}[tbp]
\centering
\includegraphics[width=0.48\columnwidth]{chapter5/figs/phsp_qsq_dist.pdf}
\caption[  The \qsq distribution of simulated \BdToKstmm events.  ]
{ The \qsq distribution of simulated \BdToKstmm events 
generated using a phase space model. The phase space available 
for the decay decreases towards high \qsq. The distribution of events is uniform in \ctl, \ctk and $\phi$.
 ~\label{fig:phspqsq} } 
\end{figure}
This sample of simulated events was used to calculate the efficiency to correct for the acceptance effects, as described in Sec.~\ref{sec:kstmm:ac}.

\subsubsection{MC11}

The samples of MC11 that were used in the second angular analysis of \BdToKstmm
consist of several signal decay modes, including \BdToKstmm, \BdToJpsiKstar, \BuToKmm, \BuToKstmm, \BsToPhimm and \LbToLmm.
The \B decays  were generated using the \verb=BTOSLLBALL=~\cite{PhysRevD.61.074024} model from \evtgen~\cite{Lange:2001uf}
to model the \bquark\to\squark FCNC decay.
This model calculates the helicity amplitudes for the \Bd\to\Kstarz transition using the form factors
calculated with the QCD sum rule using Standard Model parameters.
For the generation of the non-\Bd modes decays, the same model is used based on the assumption
 that the masses and kinematic distributions of the parent and daughter particles are approximately equal.
The \Lb decay was generated uniformly in phase space.
The \BdToJpsiKstar simulation was also used to test the corrections applied to the phase space \BdToKstmm sample
 to make the data match the simulation and determine the accuracy of any corrections applied.
All of the samples were used to determine the level of irreducible `peaking' background decays 
that satisfy all the selection criteria and may introduce a systematic bias. 

\subsubsection{Data-simulation validation}

The complete set of data-simulation corrections described previously were verified by applying the procedure to simulated \BdToJpsiKstar candidates. 
The distribution of the BDT response of the \BdToJpsiKstar candidates in data and simulation is given in Fig~\ref{fig:bdtcomparison}.
\begin{figure}[tbp]
\centering
\subfigure[BDT distribution]{\includegraphics[width=0.48\columnwidth]{appendix/figs/comp_jpsikstar_default_BDT.pdf}}
\caption[The distribution of MVA classification values for \Bd\to\jpsi\Kstarz candidates from data and simulation.    ]
{The distribution of MVA classification values for \Bd\to\jpsi\Kstarz candidates from data and simulation.
The MVA is described in Sec~\ref{sec:kstmm:sel}. The data (black) is from the 1.0\invfb sample. 
The corrected (\textcolor{red}{red}) and uncorrected (\textcolor{blue}{blue}) 
 \BdToJpsiKstar candidates are from the same MC11 simulation sample.
~\label{fig:bdtcomparison}}
\end{figure}
There is good agreement between the data and the corrected-simulated candidates, 
giving confidence that the set of corrections replicates the BDT selection efficiency correctly.





\input{chapter5/selection}

\input{chapter5/acceptance_correction}

\section{Angular analysis}
\label{sec:kstmm:pdf}

Each of the angular analyses were performed by simultaneously fitting a \PDF for the mass and the angular distribution 
to the data. % the extended maximum likelihood fit method.
%The \PDF used is a combination of a model for the the \Bd mass spectrum and the angular \PDF.
The simultaneous fit to the \Bd mass spectrum and to the angles ensures that the
 maximum information available is used to reduce the error on all of the angular observables.
It also ensures that the correlations are propagated correctly between the angular observables.
The total \PDF ($F$) is a combination of a model for the signal ($S$) and background ($B$) ,
each containing component \PDFs to describe the mass distribution and the angular distribution,
\begin{align}
F(\mB, \ctl, \ctk, \phi) = & f_{sig} \left( S_{i}(\mB) \times S_{i}(\ctl,\ctk,\phi) \right) \nonumber \\ 
&+ ( 1 - f_{sig} ) \left( B_{i}(\mB) \times B_{i}(\ctl,\ctk,\phi) \right) \, .
\end{align}
where $i$ indicates the model used for the first or second analysis.
The different components of the total \PDF are described in detail below.

The dataset is divided into seven bins of \qsq.
 There are six separate bins in the full \qsq range, detailed in Table~\ref{tbl:qsqbins}.
\begin{table}
\centering
\caption[ The \qsq binning scheme used in both angular analyses. ]
{ The \qsq binning scheme used in both angular analyses. 
The binning is analogous to the binning used in Ref~\cite{PhysRevLett.103.171801} including
 the  \qsq region of 1 to 6 \gevgevcccc .~\label{tbl:qsqbins} }
\begin{tabular}{|c|c|}
\hline
lower limit (\gevgevcccc)  & upper limit (\gevgevcccc) \\
\hline
0.1 & 2 \\
2 & 4.3 \\
4.3 & 8.68 \\
10.09 & 12.9 \\
14.18 & 16 \\
16 & 19 \\
\hline
1.0 & 6.0 \\
\hline
\end{tabular}
\end{table}
The binning is analogous to the binning used in Ref~\cite{PhysRevLett.103.171801} along with the
 region from $1 < \qsq < 6 \gevgevcccc $, which is a theoretically clean region where the observables are easily calculable.
The binning is chosen such that there is a bin below and above the point where \AFB is
 predicted to change sign in the Standard Model, and also with boundaries to avoid the \ccbar resonances.

\subsection{Mass model}
\label{sec:kstmm:massmodel}

Two different mass models were used to parametrise the \Bd signal invariant mass distribution.
The signal invariant mass model used to parametrise the \Bd invariant mass distribution in the first angular analysis was a double Gaussian function,
\begin{align}
\label{eq:totpdf}
S_{(1)}\left( \mkpimm ; \sigma_1, \sigma_2, \alpha, n \right) = & \, f \times G\left(  \mkpimm  ; m_B , \sigma_1 \right) \nonumber\\
& + \left(1-f\right) \times G\left(  \mkpimm ; m_B , \sigma_2\right) \, ,
\end{align}
where $f$ is the fraction of signal between each component and $\sigma_{1,2}$ are the different widths of each Gaussian component.
The signal mass model for the second angular analysis was an empirical model consisting of two Crystal Ball functions.
The Crystal Ball was a function developed to model the radiative 
tail from the \bbbar resonances~\cite{Skwarnicki:1986xj}.
It consists of a Gaussian distribution with an 
exponential tail and is expressed for a given mass ( \mkpimm ) as
\begin{equation}
\CB \left(  \mkpimm ; \mB , \sigma, \alpha, n \right) = N
\begin{cases}
  \exp{\left(\frac{-(\mkpimm - \mB )^2 }{2\sigma^2}\right)} 
 & \text{if} \,\, \mkpimm >  \alpha  \\
 \frac{\left(\frac{n}{\alpha^2}\right)^n}{\frac{\left(\mkpimm-\mB \right)}{\sigma} + \frac{n}{\alpha} - \alpha }
 & \text{if} \,\, \mkpimm\leq  \alpha
\end{cases}
\end{equation}
where $N$ is the signal normalisation, \mB is the nominal B mass, $\sigma$ is the Gaussian width and $n$ and $\alpha$ are the tail parameters.
Here the Crystal Ball function is used as an empirical formula to describe tails in the \Bd mass spectrum from resolution effects.
The parameters for the \Bd mass signal shape are assumed to be equivalent for both Crystal Ball functions except for the widths,
\begin{align}
S_{(2)}\left( \mkpimm ; \sigma_1, \sigma_2, \alpha, n \right) = & 
\, f \, \CB\left(  \mkpimm  ; m_B , \sigma_1, \alpha, n \right) \nonumber\\
& + \left(1-f\right) \times \CB\left(  \mkpimm ; m_B , \sigma_2, a, n \right) \ .
\end{align}

The shape of the signal mass model for both analyses is taken from fits to the \BdToJpsiKstar invariant mass spectrum. 
Due to the high statistics of \BdToJpsiKstar in the data, it is necessary to include an 
 additional contribution from the suppressed \BsToJpsiKstar mode.
The decay \BsToJpsiKstar is suppressed by a factor of \fd\Vtd/\Vts\fs compared to \BdToJpsiKstar.
The model used for the \BsToJpsiKstar is identical to the model for the \BdToJpsiKstar except for the central 
mass value and shares all of it's parameters with the \BdToJpsiKstar model. 
The only remaining free parameter is the relative normalisation between the two contributions.
There is a relative factor
\begin{align}
\frac{n_{\Bd}}{n_{\Bs}} =  0.007\pm0.002 \, ,
\end{align}
which is applied as a Gaussian constraint on the overall size of the \BsToJpsiKstar contribution.

The model for the background contribution to the \mkpimm spectrum for both analyses is the same. This is an exponential function,
\begin{align}
B_{(1,2)}\left(\mkpimm;\lambda\right) = N_B \exp{\left(-\lambda\mkpimm\right)} \, ,
\end{align}
where $\lambda$ is the decay constant for the exponential and $N_B$ is the normalisation of the background \PDF.

\subsection{Angular model}
\label{sec:kstmm:angularmodel}

The signal angular model for each of the analyses is a simplification of the full angular distribution for \BdToKstll 
as described in Sec.~\ref{sec:fullangdist}.
The angular distribution is integrated over one bin of \psq and integrated over each of the six bins of \qsq.
The signal model used in the 0.38\invfb angular analysis to measure \AFB and \FL is 
%a function of the angles \ctl and \ctk and integrated over $\phi$.% containing \AFB and \FL. 
%Integrating Eq.~\ref{eq:theo5d} over \psq and \varphi gives
\begin{align}
S_{(1)}(\ctl,\ctk) = & \frac{9}{16}   \Bigg( 2 \FL \ctksq ( 1 - \ctlsq )   \nonumber\\ 
& + \frac{1}{2} ( 1 - \FL ) ( 1 - \ctksq ) ( 1 + \ctlsq )  \nonumber\\
& +  \frac{4}{3} \AFB ( 1 - \ctksq ) \ctl   \Bigg) . 
\end{align}
The angular distribution for the 1.0\invfb angular analysis was extended to include angular observables dependent on \varphi. 
The distribution was simplified using the transformation described in Sec.~\ref{sec:fullangdist} and~\cite{Ksteepubnote}.
The analysis uses two parameterisations  of the angular distribution.
%, the first of which contains the regular angular observables from~\cite{AltmannshoferBall,Kruger:2005ep}.
The angular distribution for \AFB, \FL, \OS3 and \OS9 is given by
\begin{align}
\label{eq:theo4doriginal}
S_{(2a)}(\ctl,\ctk,\phiprime) =   \frac{9}{16\pi}  & \Bigg(  2 \FL \ctksq ( 1 - \ctlsq )  \nonumber \\ 
&  + \frac{1}{2} ( 1 - \FL ) ( 1 - \ctksq ) ( 1 + \ctlsq )  \nonumber \\
&  + \OS3 ( 1 - \ctksq ) (  1 - \ctlsq ) \cos2\phiprime  \nonumber \\
&  +  \frac{4}{3} \AFB ( 1 - \ctksq ) \ctl  \nonumber \\ 
&  + \OS9 ( 1 - \ctksq ) ( 1 - \ctlsq) \sin2\phiprime  \Bigg) . 
\end{align}
The re-parametrised angular distribution contains the transverse angular observables (\ATRe, \AT2, \ATIm) as described in Sec.~\ref{sec:kstmm:obs},
\begin{align}
\label{eq:theo4dreparam}
S_{(2b)}(\ctl,\ctk,\phiprime) =  \frac{9}{16\pi}  & \Bigg(  2 \FL \ctksq ( 1 - \ctlsq )  \nonumber \\ 
&  + \frac{1}{2} ( 1 - \FL ) ( 1 - \ctksq ) ( 1 + \ctlsq )  \nonumber \\
&  + \frac{1}{2} ( 1 - \FL ) \AT2 ( 1 - \ctksq ) (  1 - \ctlsq ) \cos2\phiprime  \nonumber \\
& +  \frac{4}{3}  ( 1 - \FL ) \ATRe ( 1 - \ctksq ) \ctl  \nonumber \\ 
& +  ( 1 - \FL ) \ATIm ( 1 - \ctksq ) ( 1 - \ctlsq) \sin2\phiprime  \Bigg) . 
\end{align}
The angular distribution used to measure \OA9 uses the CP anti-symmetric definition of \varphi where the sign changes
for \Bd and \Bdb decays as given in Sec~\ref{sec:kstmm:obs}.
The signal angular distribution is a function of \phiprimecp,
\begin{align}
\label{eq:theo4dphi}
S_{(2c)}(\ctl,\ctk,\phiprimecp) =   \frac{9}{16\pi}  & \Bigg(  2 \FL \ctksq ( 1 - \ctlsq )  \nonumber \\ 
&  + \frac{1}{2} ( 1 - \FL ) ( 1 - \ctksq ) ( 1 + \ctlsq )  \nonumber \\
&  + \OA3 ( 1 - \ctksq ) (  1 - \ctlsq ) \cos2\phiprimecp  \nonumber \\
&  +  \frac{4}{3} \AFB ( 1 - \ctksq ) \ctl  \nonumber \\ 
&  + \OA9 ( 1 - \ctksq ) ( 1 - \ctlsq) \sin2\phiprimecp  \Bigg) . 
\end{align}

The model for the background in each of the angles is equivalent for both angular analyses.
The background \PDF is an $n^{\text{th}}$ order Chebychev polynomial of the first kind for each angle,
\begin{align}
T_{n}(x) = \cos\left(n\arccos(x)\right) \, .
\end{align}
The total background angular \PDF is factorised into each of the angles,
\begin{align}
B(\mkpimm) = P`^{bkg}_{n}(\ctl, \ctk, \phiprime) = P_{n}^{L}(\ctl) \times P_{n}^{K}(\ctk) \times P_{n}^{P}\phiprime) \, .
\end{align}
The assumption that the background angular distribution factorises was tested using the point-to-point dissimilarity test~\cite{Williams:2010vh}.
The probability of the test statistic having a value less than the test statistic of the data was 25\%.
This value is entirely compatible with the assumption that the background factorises into the three angles.

\subsection{Result extraction}
\label{sec:kstmm:resextraction}

The signal \PDF is fitted to the data by performing an unbinned maximum-log-likelihood fit to the data,
minimising
\begin{align}
 - \log \mathcal{L} = \sum_{i}^{N}  \omega_i F(\mkpimm^i, \ctl^i, \ctk^i, \phi^i,  \vec{p} , \vec{O} ) ,
\end{align}
where $F$ is the total \PDF described in Eq.~\ref{eq:totpdf}. % which in terms of the \kpimm invariant mass and angles of each of the candidates.
The set of parameters for the signal and background mass models are $\vec{p}$, while $\vec{O}$ is the set of angular observables.
Each of the data candidates is weighted for acceptance as described in Section~\ref{sec:kstmm:ac}.
These weights distort the shape of the likelihood such that the errors extracted from the standard NLL minimisation are not 
guaranteed to be the true errors.
In each angular analysis, two different techniques were used to extract the likelihood minima and a better estimate of the error from the likelihood function.
In the 0.38\invfb analysis, the profile likelihood was calculated and the error determined from the two-dimensional 68\% confidence interval in both \AFB and \FL.
For the 1.0\invfb analysis, the errors were extracted in a Frequentist manner using the Feldman-Cousins (FC) technique~\cite{PhysRevD.57.3873}.

The FC technique maps out the likelihood for an observable, allowing the size of the confidence intervals for a given observable to be calculated. 
For an observable of interest in a given set of parameters, the ratio between the likelihood calculated with all parameters free $(\mathcal{L}_0)$ and the likelihood calculated with the observable fixed is calculated $(\mathcal{L}_1)$.
The ratio between these likelihood ($R_{data}$) is obtained for the result obtained from data, and for a large ensemble of toy datasets ($R_i$).
The fraction of $R_i < R_{data}$ ($f_R$) is proportional to the probability of the data result being the most optimum solution in the phase space of the parameters.
This fraction is calculated for a range of values for the observable and the 68\% confidence limits on the observable are calculated from the points where the $f_R<0.68$.

The results of the angular fits along with the calculated confidence limits are shown in Section~\ref{sec:kstmm:res}.



\section{Systematic uncertainties}
\label{sec:swave:meas:sys}

There are several distinct sources of systematic uncertainty that are
 considered to affect the measurement of the \kpi S-wave in \BdToKpimm.
The systematic uncertainties affecting the event selection, the corrections to simulation and the model to describe the
\B mass distribution have previously been considered in Section~\ref{sec:kstmm:sys}.
These effects have a possible impact in this analysis and are therefore tested in a similar manner.
There are new sources of systematic uncertainty from the model used for the \mkpi distribution.
These come from both the background and signal models, along with the phase space integration
 assumed over \qsq.
 Each of the sources of systematic uncertainty are discussed below along with the method used to 
 estimate a possible bias.

\subsubsection{The selection}

The possible mis identification of \Kstar and \Kstarb was found in Chapter~\ref{chap:kstmm}  
to be negligible and consequently can be ignored for this measurement.
The amount of \kpi swaps should be negligible due to the cut placed on the \kdllkpi and \pidllkpi combination and is ignored.
The contributions from possible peaking background decays are vetoed to a sufficient degree and similarly ignored.
%however, this cut was varied by so-and-so in order to understand the effect of XXX

\subsubsection{The data-simulation corrections}

The sources of systematic uncertainty that contribute to the corrections applied to the simulation are described in Section~\ref{sec:kstmm:sys}. 
These come from the relative efficiency between the data and the simulation for the tracking, the trigger and the muon identification.
 The smearing of the track IP and the regeneration of the hadron \dll distributions are also possible sources of systematic uncertainty.
The degree of systematic uncertainty contributed by these corrections is evaluated using the same method as described in Section~\ref{sec:kstmm:sys}.

\subsubsection{The acceptance correction}

The factorisation of the efficiency between \psq and \ctk is tested by reducing the range of \ctk.
This removes the contribution from events at high \ctk which may have an erroneous weight applied from the acceptance correction.
The reduction in the range of \ctk changes the model used for the signal since the integral no longer vanishes.
The integral over the angular distribution in terms of \psq, \qsq and \ctk for a symmetric \ctk range is given by
\begin{align}
\int_{-c}^{c} &S(\mkpi,\ctk;\FSi,\ASi,\FL) \,\dctk \nonumber\\
 &= \frac{1}{2} \FSi(\mkpi) (2c) + \FPi(\mkpi) \bigg[ c^3 \FL + \frac{3}{4} ( 1 - \FL ) ( 2c - \frac{2}{3} c^3 ) \bigg]  \, ,
\end{align}
where the term with \ASi vanishes for the symmetric \ctk range. For a \ctk range of less than $-1$ to $1$, 
\FL does not integrate out and a correction is required.

The change in \FS in the \qsq bin from 1 to 6 \gevgevcccc when the \ctk range is changed is shown in Fig.~\ref{fig:fs:ctkrange}.
\begin{figure}[tbp]
\centering
\includegraphics[width=0.48\columnwidth]{chapter7/figs/fiducial_cuts_FS_0.pdf}
\caption[ The change in \FS for different ranges of \ctk.   ]
{The change in \FS for different ranges of \ctk. 
The effect of including events at high \ctk  may contribute to a source of systematic uncertainty.
The change in \FS when only \psq and \qsq is fitted is shown in black and 
the change in \FS when \ctk is included in the angular distribution is shown in \textcolor{red}{red}.
\label{fig:fs:ctkrange} }
\end{figure}
To calculate the correction from the integration, the values of \FL measured in Chapter~\ref{chap:kstmm} are used.
It is possible to see that the results with and without fitting \ctk for different fiducial ranges of \ctk are compatible.
The contribution from this source of systematic uncertainty is chosen to be from the change in \FS when the integration range 
is changed from  $|\ctk|<1$ to $|\ctk|<0.7$.
%Within $|\ctk|<0.7$, the acceptance is flat as seen in Fig.~\ref{fig:fs:ctkrange}.

The integration over \ctk is also checked by fitting the angular distribution in terms of \psq and \ctk.
The acceptance correction from Sec.~\ref{sec:kstmm:ac} is used as an approximation.
The values of \FL obtained from these fits are compatible within statistical errors with the results obtained in Section~\ref{sec:kstmm:res}.
%No systematic uncertainty is assigned from the approximation of the angular distribution by ignoring \ctl and $\phi$.

\subsubsection{The fit model}

There are several possible sources of systematic uncertainty in the choice of model used to describe the \kpimm mass distribution.
The degree to which is \FS  can be affected by the \mB mass distribution comes from the two-stage fit used to obtain the overall fraction of signal in the data for a given region of \qsq.
 In order to test for any bias in the signal shape, the double Crystal Ball function was replaced by a double Gaussian function.
 This will change the tails of the signal distribution and change the quality of the background fit.
In order to test possible uncertainties from the choice of background model, the exponential function was replaced with a second order Chebychev polynomial as the background model.

Following the work in Section~\ref{sec:swave:isobar}, an isobar model consisting of 
a constant non-resonant term, the $\Kstarz(1430)$ and the $\kappa(600)$ was used as an alternative model
the \kpi mass spectrum.

%\subsubsection{uncertainties on the \mkpi model}
%There is a small degree of uncertainty on both the choice of the model for the signal S-wave and the model for the \kpi background.
%[CONCLUSOION].\\
%The use of Equation~\ref{eq:swave:mkpi:background} to model the background \kpi mass spectrum was checked by 
%XXXXXXXXX.
%[CONCLUSOION].\\

The degree to which the approximation of the phase space factor across the \qsq bin effects the final value of \FS is evaluated by using
the \qsq values at both the low and high edge of the \qsq bin in the phase space factor,
\begin{align}
\rho( \mB, \psq,\qsq ) = \left\{ \begin{matrix} \rho( \mB, \psq,\qsq_{max} ) \\\rho( \mB, \psq,\qsq_{min} )   \end{matrix} \right.
\end{align}
This was found to contribute to a minor source of systematic uncertainty.

\subsection{Summary of systematic uncertainties}

The size of possible contributions from sources of systematic uncertainty on the measurement of \FS are given in Table~\ref{tbl:swave:meas:fs:results}.
\begin{landscape}{\scriptsize
\begin{table}[tbp]
\centering
\caption[  Table of possible sources of systematic uncertainty on the measurement of \FS.  ]
{ Table of possible sources of systematic uncertainty on the measurement of \FS. 
Only the bins with a non-zero S-wave contribution are shown.
The bins with no S-wave are unaffected by the possible sources of systematic uncertainty.
 ~\label{tbl:swave:meas:fs:results} }
\begin{tabular}{|c|c|c|c|c|c|c|c|c|}
\hline
\qsq (\gevgevcccc) bin & $0.1<\qsq<2.0$&$2.0<\qsq<4.3$&$4.3<\qsq8.68$&$10.09<\qsq<12.9$&$14.18<\qsq16.0$&$16<\qsq<19$&$1<\qsq<6$\\
Central value & 0.164 & 0.000 & 0.092 & 0.001 & 0.000 & 0.000 & 0.083\\ 
\hline
Stat down & 0.060 & 0.000 & 0.033 & 0.000 & 0.000 & 0.000 & 0.048\\ 
Stat up & 0.069 & 0.135 & 0.039 & 0.044 & 0.007 & 0.002 & 0.057\\ 
\hline
Sys down & 0.013 & 0.000 & 0.046 & 0.000 & 0.000 & 0.000 & 0.050\\ 
Sys up & 0.011 & 0.001 & 0.021 & 0.000 & 0.000 & 0.000 & 0.018\\ 
\hline
Bkg order 1 & 0.000 & 0.000 & 0.000 & 0.000 & 0.000 & 0.000 & -0.026\\ 
Bkg order 3 & 0.000 & 0.000 & 0.000 & 0.000 & 0.000 & 0.000 & 0.000\\ 
PID down  & 0.000 & 0.000 & 0.000 & 0.000 & 0.000 & 0.000 & 0.000\\ 
PID up & 0.000 & 0.000 & 0.000 & 0.000 & 0.000 & 0.000 & 0.000\\ 
fit the \B mass & 0.000 & 0.000 & -0.001 & 0.000 & 0.000 & 0.000 & 0.003\\ 
isobar model & 0.000 & 0.000 & 0.000 & 0.000 & 0.000 & 0.000 & 0.000\\ 
Muon ID down & 0.000 & 0.000 & -0.029 & 0.000 & 0.000 & 0.000 & 0.000\\ 
Muon ID up & 0.000 & 0.000 & 0.000 & 0.000 & 0.000 & 0.000 & 0.000\\ 
IP smearing & 0.000 & 0.000 & -0.029 & 0.000 & 0.000 & 0.000 & 0.000\\ 
\qsq efficiency  down & 0.000 & 0.000 & 0.000 & 0.000 & 0.000 & 0.000 & 0.000\\ 
\qsq efficiency up & 0.000 & 0.000 & 0.000 & 0.000 & 0.000 & 0.000 & 0.000\\ 
\ctk limit & -0.008 & 0.001 & 0.001 & 0.000 & 0.000 & 0.000 & -0.029\\ 
\qsq high edge & -0.010 & 0.000 & -0.020 & 0.000 & 0.000 & 0.000 & -0.018\\ 
\qsq low edge & 0.011 & 0.000 & 0.021 & 0.000 & 0.000 & 0.000 & 0.018\\ 
Tracking Down & 0.000 & 0.000 & 0.000 & 0.000 & 0.000 & 0.000 & 0.000\\ 
Tracking Up & 0.000 & 0.000 & 0.000 & 0.000 & 0.000 & 0.000 & 0.000\\ 
Trigger Down & 0.000 & 0.000 & 0.000 & 0.000 & 0.000 & 0.000 & -0.026\\ 
Trigger Up & 0.000 & 0.000 & 0.000 & 0.000 & 0.000 & 0.000 & 0.000\\ 
\hline
\end{tabular}
\end{table}}
\end{landscape}
The dominant sources of systematic uncertainty come from the use of the crystal ball function to fit the \kpimm invariant mass
and from the mis-modelling of the \qsq and \psq efficiency.






\section{Results}
\label{sec:swave:meas:res}

The results of the fit to each of the \qsq bins for the \Bd mass spectrum are shown in Fig.~\ref{fig:swave:meas:fits:1} and for the \kpi mass spectrum are shown in Fig.~\ref{fig:swave:meas:fits:2}.
\begin{figure}[tbp]
\centering
\subfigure[$0.10<\qsq<2.00\gevgevcccc$]{\includegraphics[width=0.45\columnwidth]{chapter7/figs/fits/fit_kstarmumu_swave_mkpi_range_lass_mass_canvas_0.pdf}}
\subfigure[$2.00<\qsq<4.30\gevgevcccc$]{\includegraphics[width=0.45\columnwidth]{chapter7/figs/fits/fit_kstarmumu_swave_mkpi_range_lass_mass_canvas_1.pdf}}
\subfigure[$4.30<\qsq<8.68\gevgevcccc$]{\includegraphics[width=0.45\columnwidth]{chapter7/figs/fits/fit_kstarmumu_swave_mkpi_range_lass_mass_canvas_2.pdf}}
\subfigure[$10.09<\qsq<12.9\gevgevcccc$]{\includegraphics[width=0.48\columnwidth]{chapter7/figs/fits/fit_kstarmumu_swave_mkpi_range_lass_mass_canvas_3.pdf}}
\subfigure[$14.18<\qsq<16.0\gevgevcccc$]{\includegraphics[width=0.45\columnwidth]{chapter7/figs/fits/fit_kstarmumu_swave_mkpi_range_lass_mass_canvas_4.pdf}}
\subfigure[$16.00<\qsq<19.0\gevgevcccc$]{\includegraphics[width=0.45\columnwidth]{chapter7/figs/fits/fit_kstarmumu_swave_mkpi_range_lass_mass_canvas_5.pdf}}
\subfigure[$1.00<\qsq<6.00\gevgevcccc$]{\includegraphics[width=0.45\columnwidth]{chapter7/figs/fits/fit_kstarmumu_swave_mkpi_range_lass_mass_canvas_6.pdf}}
\caption{ The result of the fit to the \kpimm mass spectrum in six \qsq bins for selected \BdToKpimm events from 1.0\invfb of data. ~\label{fig:swave:meas:fits:1} }
\end{figure}
\begin{figure}[tbp]
\centering
\subfigure[$0.10<\qsq<2.00\gevgevcccc$]{\includegraphics[width=0.45\columnwidth]{chapter7/figs/fits/fit_kstarmumu_swave_mkpi_range_lass_mkpi_canvas_0.pdf}}
\subfigure[$2.00<\qsq<4.30\gevgevcccc$]{\includegraphics[width=0.45\columnwidth]{chapter7/figs/fits/fit_kstarmumu_swave_mkpi_range_lass_mkpi_canvas_1.pdf}}
\subfigure[$4.30<\qsq<8.68\gevgevcccc$]{\includegraphics[width=0.45\columnwidth]{chapter7/figs/fits/fit_kstarmumu_swave_mkpi_range_lass_mkpi_canvas_2.pdf}}
\subfigure[$10.09<\qsq<12.9\gevgevcccc$]{\includegraphics[width=0.45\columnwidth]{chapter7/figs/fits/fit_kstarmumu_swave_mkpi_range_lass_mkpi_canvas_3.pdf}}
\subfigure[$14.18<\qsq<16.0\gevgevcccc$]{\includegraphics[width=0.45\columnwidth]{chapter7/figs/fits/fit_kstarmumu_swave_mkpi_range_lass_mkpi_canvas_4.pdf}}
\subfigure[$16.0<\qsq<19.0\gevgevcccc$]{\includegraphics[width=0.45\columnwidth]{chapter7/figs/fits/fit_kstarmumu_swave_mkpi_range_lass_mkpi_canvas_5.pdf}}
\subfigure[$1.00<\qsq<6.00\gevgevcccc$]{\includegraphics[width=0.45\columnwidth]{chapter7/figs/fits/fit_kstarmumu_swave_mkpi_range_lass_mkpi_canvas_6.pdf}}
\caption{ The results of the fit to the \kpi mass spectrum for selected \BdToKpimm events from 1.0\invfb of data. ~\label{fig:swave:meas:fits:2} }
\end{figure}
The results of the measurement of the \kpi S-wave in \BdToKpimm using 1.0\invfb of integrated luminosity collected at \lhcb are presented in Fig.~\ref{fig:swave:meas:results}.
\begin{figure}[tbp]
\centering
\includegraphics[width=0.66\columnwidth]{chapter7/figs/fits/plot_swave_FS.pdf}
\caption[ The fraction of \kpi S-wave in six bins of \qsq for selected \BdToKpimm 
events from 1.0\invfb of data collected at $\sqs=7\tev$ at \lhcb in 2011.   ]
{ The fraction of \kpi S-wave in six bins of \qsq for selected \BdToKpimm 
events from 1.0\invfb of data collected at $\sqs=7\tev$ at \lhcb in 2011.
For the regions where no S-wave is found, the upper error bar indicates the 95\% confidence limit.
~\label{fig:swave:meas:results} }
\end{figure}
The central values, statistical and systematic errors in 6 bins of \qsq are given in Table~\ref{tbl:swave:results}.
\setlength\extrarowheight{3pt}
\begin{table}[tbp]
\centering
\caption[ Table showing the fraction of \kpi S-wave in six bins of \qsq for selected \BdToKpimm 
events from 1.0\invfb of data collected at $\sqs=7\tev$ at \lhcb in 2011.    ]
{ Table showing the fraction of \kpi S-wave between 0.64<\psq<1.00 in six bins of \qsq for selected \BdToKpimm 
events from 1.0\invfb of data collected at $\sqs=7\tev$ at \lhcb in 2011.
For the regions where no S-wave is found, results are quoted at 95\% confidence limit.
~\label{tbl:swave:results} }
\begin{tabular}{|c|c|}
\hline
Bin (\gevgevcccc) & \FS \\ 
\hline
 $ 0.10<\qsq< 2.00$   & $0.164_{-0.060}^{-0.069}~_{+0.013}^{-0.011} $\\ 
 $ 2.00 <\qsq<  4.30$  & $ < 0.135~(\mathrm{at~95\%~C.L.}) $\\ 
 $ 4.30 <\qsq<  8.68$  & $0.092_{-0.033}^{-0.039}~_{+0.046}^{-0.021} $\\ 
 $ 10.09 <\qsq<  12.90$   & $ < 0.044~(\mathrm{at~95\%~C.L.}) $\\ 
 $ 14.18 <\qsq<  16.00$  & $ < 0.007~(\mathrm{at~95\%~C.L.}) $\\ 
 $ 16.00 <\qsq<  19.00$  & $ < 0.002~(\mathrm{at~95\%~C.L.}) $\\ 
 $ 1.00 <\qsq<  6.00$   & $0.083_{-0.048}^{-0.057}~_{+0.050}^{-0.018} $\\ 
\hline
\end{tabular}
\end{table}
There is an indication of a non-zero S-wave contribution at low \qsq, specifically in the region below 2\gevgevcccc, the region from 4.3 to 8.68\gevgevcccc and in the region from 1 to 6 \gevgevcccc.
The $p$-values of the zero S-wave hypothesis for each of the bins with non-zero S-wave are 0.05, 0.07 and 0.02 respectively. 
None of these bins are significant enough to provide evidence of a \kpi S-wave and the other bins contain insufficient events to measure any contribution from a \kpi S-wave.
The value of \FS in the \qsq bin from 1 to 6 \gevgevcccc and in the \psq bin from 0.64 to 1 \gevgevcccc was found to be
\begin{align}
\FS = 0.083^{+0.057}_{-0.048}(\mathrm{stat.})^{+0.018}_{-0.050} (\mathrm{syst.})
\end{align}



%\input{chapter5/theoint}


\section{Conclusions}

The angular analysis of \BdToKstmm  at \lhcb was performed on 
both 0.38\invfb and 1.0\invfb of data taken in 2011. 

Clean samples of \BdToKstmm candidates were selected using both a cut-based selection and a multi-variate algorithm. 
There were around 340 candidates in 0.38\invfb and 900 candidates in 1.0\invfb of data.
The first dataset is comparable to previous results from the \babar, \belle and \cdf 
and the second dataset is the largest sample of \BdToKstmm candidates at one experiment to date.

The candidates were corrected for the acceptance effect introduced by the reconstruction and selection by applying a weight to each event.
The first analysis used a $k$-nearest-neighbour method to calculate the efficiency to selected simulated events at a point in \ctl and \ctk.
This was an accurate calculation but the prevision and accuracy was limited by the number of simulated candidates in the regions of phase space with low efficiency.
The second analysis calculated the efficiency from a function fitted to each of the \ctl, \ctk and \varphi distributions independently.
This limitation of this method is that it assumes that the efficiency factorises into each dimension, which introduces additional sources of systematic uncertainty.
The method of weighting each of the \BdToKstmm candidate for their acceptance was chosen over the alternative method,
 of combining the signal model and a function for the efficiency, in order to minimise the number of free parameters in the final model. 	
This allowed measurement of the angular observables using the multi-dimensional angular distribution which fully incorporated the correlations between the angular observables.

The acceptance effect and the data-simulation corrections were the dominant sources of systematic uncertainty for both analyses. 
This is because the events with lowest efficiency, at extreme \ctl and high \ctk, have a large effect on the central value of \AFB and \FL.
This can be improved by a better understanding of the simulation and the efficiency to select \BdToKstmm candidates
 but the understanding of the efficiency in this region of phase space is a limitation on the accuracy of the measurement.

The first angular analysis obtained the worlds most precise measurements of the observables \AFB and \FL as well as measuring the differential branching fraction.
The second angular analysis improved the measurements of \AFB and \FL 
as well as measuring several new angular observables for the first time.
The measurements of \AFB and \FL from \lhcb along with the measurements from \babar, \belle and \cdf are shown in Figure~\ref{fig:res:comboexp}.
\begin{figure}[tbp]
\centering
\subfigure[]{\includegraphics[width=0.48\columnwidth]{chapter5/figs/plot_FLExp.pdf}}
\subfigure[]{\includegraphics[width=0.48\columnwidth]{chapter5/figs/plot_AFBExp.pdf}}
\caption{The measurements of the angular observables \FL and \AFB from \lhcb, \babar~\cite{Aubert:2007hz,Aubert:2008ju}, 
\belle~\cite{PhysRevLett.103.171801}  and \cdf~\cite{Aaltonen:2011cn,Aaltonen:2011ja} along with the theoretical prediction from Ref.~\cite{Bobeth:2011gi}. 
It is possible to see that the \lhcb results are the most precise and are compatible with the SM prediction.~\label{fig:res:comboexp}}
\end{figure}

The combination of these results, along with other radiative,
semi-leptonic and purely leptonic decays has enabled stringent limits to be set on the 
values for the Wilson coefficients \C7, \C9 and \C10 along with 
a high limit on the mass scale of any particle that contributes via
electroweak penguin diagrams~\cite{Bobeth:2011nj,Altmannshofer:2011gn,Altmannshofer:2012az}.
These constraints affect any new physics model that contains high mass particles with flavour couplings, 
providing a model-independent test of the mass scale of contributions from physics beyond the standard model.









