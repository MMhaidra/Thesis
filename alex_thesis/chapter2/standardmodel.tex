\section{ The Standard Model }
\label{sec:sm:sm}

The Standard Model is a quantum field theory for the fermion fields described by the gauge group
 $\grpsuthree_{\mathrm{C}} \otimes\grpsutw_{\mathrm{L}} \otimes \grpuone_{\mathrm{Y}}$~\cite{Yang:1954ek}.
The subgroup  $\grpsuthree_{\mathrm{C}}$ describes the quarks and the strong interactions, 
and the subgroup $\grpsutw_{\mathrm{L}} \otimes \grpuone_{\mathrm{Y}}$ unifies the electromagnetic and weak interactions.
These groups are required to be locally gauge invariant which leads to the addition of 
fields representing the bosons.
The requirement of local gauge invariance implies massless fermion fields but the Higgs mechanism
provides a solution by spontaneously breaking the electroweak gauge group
to distinguish the weak and electromagnetic interactions and provides locally gauge invariant
 mass terms to each particle.

In the SM, the fermions are divided up into three generations with two quarks and two leptons per generation.
The quarks and charged leptons in each generation are more massive than the previous generation but are otherwise identical.
The structure of the particles in the SM is shown in Table~\ref{tbl:sm}.
\begin{table}[tbp]
\centering
\caption{ Table of particles in the SM ~\label{tbl:sm}} 
\begin{tabular}{|c|c|c|c|}
\hline
Generation & 1 & 2 & 3 \\
\hline
Quarks &$\begin{matrix}u\\d\end{matrix}$&$\begin{matrix}c\\s\end{matrix}$&$\begin{matrix}t\\b\end{matrix}$ \\
Leptons &$\begin{matrix}e\\\nu_e\end{matrix}$&$\begin{matrix}\mu\\\nu_\mu\end{matrix}$&$\begin{matrix}\tau\\\nu_\tau\end{matrix}$ \\ 
\hline
\end{tabular}
\quad
\begin{tabular}{|c|}
\hline
Bosons \\ 
\hline 
 \g \\ 
 \H \\   
\Wpm \\
\Z  \\ 
\hline
\end{tabular}
\end{table}

The dynamics of the Standard Model particles can be described by a Lagrangian,
\begin{align}
\mathcal{L}_{\mathrm{SM}} &= \mathcal{L}_{\mathrm{EW}} + \mathcal{L}_{\mathrm{QCD}} + \mathcal{L}_{Higgs} \, ,
\end{align}
where the first component describes the electroweak sector, the second the flavour sector and the last is the additional Higgs term. 
The fermion and boson dynamics and their respective interaction with the Higgs field
can be separated out to give an alternative form of the Lagrangian,
\begin{align}
\mathcal{L}_{\mathrm{SM}} &= \mathcal{L}_{Kinetic} + \mathcal{L}_{Higgs} + \mathcal{L}_{Yukawa}  \, ,
\end{align}
where the final Yukawa term describes the coupling of the Higgs field to the matter particles.
Where not explicitly referenced, the material in this chapter can be found in any good particle physics text and 
has been drawn primarily from Refs~\cite{Abers:1973qs,Halzen:1984mc,Manohar:2000dt,Mannel:2004ce}.

\subsection{Electromagnetism}

The formalism of the fermion and boson fields under a locally gauge invariant group can be demonstrated 
by consideration of the electromagnetic field.
The kinetic part of the Lagrangian is given as 
\begin{align}
\mathcal{L} = i\bar{\psi}\gamma^\mu\partial_\mu\psi - \mu \psi\bar{\psi}
\end{align}
for particle ($\psi$) and anti-particle ($\bar{\psi}$) fields.
It is trivially invariant under gauge transformations of the first kind, or global gauge invariance.
A simple example is the group \grpuone, $\psi \to \psi^{'} = e^{i\alpha} $.
The electromagnetic field is also invariant under transformations which vary in time and space. 
These are gauge transformations of the second kind, known as local gauge invariance.
The group of transformations are defined as
\begin{align}
\psi \to \psi^{'} &= e^{i\alpha(x)} \psi \\
\bar{\psi} \to \bar{\psi}^{'} &= e^{-i\alpha(x)}\bar{\psi}
\end{align}
The kinetic energy Lagrangian is now no longer invariant because the derivative gains an extra 
term of $i\alpha(x)\psi(x)$.
Therefore, the kinetic energy term is kept invariant by introducing 
 a vector field, $A_\mu$, along with a covariant derivative 
\begin{align}
D_\mu &= \left( \partial_\mu - i e A_\mu \right) \, .
\end{align}
The covariant derivative describes interactions between the particle and the gauge boson through
a $ \bar{\psi}\gamma^\mu\psi A_\mu $ term.
The coefficient of this term gives the strength of the coupling between particles and the photon, $e$,
and the modified Lagrangian is locally gauge invariant,
\begin{align}
\mathcal{L} = i\bar{\psi}\gamma^\mu D_\mu\psi - \psi\bar{\psi} \, ,
\end{align}
where the vector field, also known as the gauge boson, transforms as 
\begin{align}
A_\mu^{'} = - \frac{1}{e} \partial_\mu\psi + A_\mu \, .  
\end{align}
The dynamics of this electromagnetic gauge boson, the photon, are given by the Lagrangian
\begin{align}
\mathcal{L} = - \frac{1}{2} F_{\mu\nu} F^{\mu\nu}
\end{align}
where the Faraday tensor is defined as 
\begin{align}
F_{\mu\nu} = \partial_\mu A_\nu - \partial_\nu A_\mu \, .
\end{align}
The complete electromagnetic Lagrangian is  
\begin{align}
\mathcal{L}_{\mathrm{EM}} =  i\bar{\psi}\gamma^\mu D_\mu\psi - \mu \psi\bar{\psi}    - \frac{1}{2} F_{\mu\nu} F^{\mu\nu}
\end{align}
and it can be seen that any additional mass term of the form $m^2 A_\mu A^\mu$ breaks the
local gauge invariance. This implies that the photon and the fermions must be massless here.

\subsection{Electroweak sector}

The theory of electroweak interactions was developed in the 1960s and provides both the weak interaction and the 
electromagnetic interaction with a mathematical basis. 
The unification of these two fundamental forces resulted in a Nobel prize for 
Glashow, Salam and Weinberg in 1979~\cite{Glashow:1961tr,Weinberg:1967tq,Salam:1968rm}.
% The electromagnetic force interacts with all known fermions and is dependent on the particle electric charge, $e$.
The weak force also interacts with all known particles and the charge associated with the weak force is called the ``weak hypercharge`.
The weak force is unique among the fundamental forces in that it is party violating.
The electroweak Lagrangian is constructed in a similar manner to the electromagnetic Lagrangian, 
\begin{align}
\mathcal{L} = \mathcal{L}_{Bosons} + \mathcal{L}_{Fermions} + \mathcal{L}_{Higgs} \, .
\end{align}
The terms for the gauge boson and terms for the fermion couplings are supplemented by the Higgs Lagrangian which 
introduces mass terms for all particles whilst remaining locally gauge invariant.
The symmetry group of the electroweak Lagrangian is  $\grpsutw_L \otimes \grpuone_Y$ where the first group 
provides the parity dependence and $Y$ is the weak hypercharge.

The fermion fields are a combination of two chiral components ($f = f_L + f_R$)
where each component can be projected out using the chiral operator,
\begin{align}
f_{\frac{R}{L}} &= \left(1\pm\gamma_5\right) f \, ,
\end{align}
that for massless particles selects each state.
The left-handed fermions are represented in \grpsutw as doublets 
\begin{align}
l_L &= \begin{pmatrix}e_L\\\nu_{e,L}\end{pmatrix},\begin{pmatrix}\mu_L\\\nu_{\mu,L} 
\end{pmatrix},\begin{pmatrix}\tau_L\\\nu_{\tau,L}\end{pmatrix},    \\
q_L &= \begin{pmatrix}u_L\\d_L\end{pmatrix},\begin{pmatrix}c_L\\s_L\end{pmatrix},\begin{pmatrix}t_L\\b_L\end{pmatrix}   
\end{align}
and the right-handed fermions are represented by \grpsutw singlets
\begin{align}
l_R &= e_R, \mu_R, \tau_R, \\
q_R &= u_R, d_R, c_R, d_R, t_R, b_R \, .
\end{align}
This incorporates the parity violating nature of the weak interactions.

There are three gauge field for \grpsutw, $W_\mu^{a=1,2,3}$ and one gauge field for  \grpuone, $B_\mu$.
The covariant derivative required to keep the Lagrangian locally invariant is
\begin{align}
D_\mu = \partial_\mu \mathbf{I} + i g_W \mathbf{T}^a W^{a}_\mu + i g_Y Y B_\mu \mathbf{I} \, ,
\end{align}
 where  $\mathbf{T}^a$ are the generators of \grpsutw which are linear combinations of the Pauli matrices,
\begin{align}
\mathbf{T}^{1,2} = \frac{1}{2}\left( \sigma^1 \pm i \sigma^2 \right) \, \text{and} \quad \mathbf{T}^3 = \sigma^3 \, .
\end{align}
In order to require that only left-handed particles participate in the weak interaction, the covariant derivative must be split into
\begin{align}
D_{L,\mu} &= \partial_\mu \mathbf{I} + i g_W \mathbf{T}^a W^{a}_\mu + i g_Y Y B_\mu \mathbf{I} \\
\mathrm{and} \quad  D_{R,\mu} &= \partial_\mu \mathbf{I} + i g_Y Y B_\mu \mathbf{I} \, .
\end{align}
The massless fermion Lagrangian is given as 
\begin{align}
\mathcal{L} = \bar{f}_{L} \left(i \gamma^\mu D_{L,\mu} \right) f + \bar{f}_{R} \left(i \gamma^\mu D_{R,\mu} \right) f_R
\end{align}
for $f = l,q$.
The Lagrangian for the dynamics of the gauge fields can be written in a 
similar manner, to the electromagnetic Lagrangian with two tensors for each symmetry group:
\begin{align}
\mathcal{L} = -\frac{1}{4} \left(  W^a_{\mu\nu} W^{a,\mu\nu} + B_{\mu\nu} B^{\mu\nu}   \right) \, ,
\end{align}
where each tensor is given through the 
\begin{align}
W^a_{\mu\nu} &= \partial_\mu W^a_{\nu}  - \partial_\nu W^a_{\mu}  + g_W \epsilon^{a,b,c} W^b_{\mu}  W^c_{\nu}  \, , \\
B_{\mu\nu} &= \partial_\mu B_\nu - \partial_\nu B_\mu \, ,
\end{align}
where $\epsilon^{a,b,c}$ are the structure constants of \grpsutw that define the relation 
\begin{align}
\tau^a = i \epsilon^{a,b,c} \left[ \tau^b, \tau^c  \right] \, .
\end{align}
%The non-Abelian nature of \grpsutw gives rise to interactions between the W bosons.

\subsubsection{Symmetry breaking}

A solution to the problem of breaking local gauge invariance by adding mass terms to the SM Lagrangian 
was proposed in Ref~\cite{Higgs:1966ev} (among others).
This involves adding a scalar field, the Higgs field ($\Phi$), to the Lagrangian of the form 
\begin{align}
\mathcal{L}_{Higgs} = (D_\mu \Phi )^{\dagger} D^\mu \Phi  + \left( \mu^2 |\Phi|^2  + \lambda |\Phi|^4 \right) \, ,
\end{align}
where the first term is the kinetic term and the second is a potential term constructed with a 
non-zero minimum in $\Phi$ at  $v = \mu/\sqrt{\lambda}$.
If the field is a scalar doublet, the minimum of the scalar field is
\begin{align}
\Phi_0 = \frac{1}{\sqrt{2}} \begin{pmatrix}0\\v\end{pmatrix} \, ,
\end{align}
which breaks the symmetry by choosing a particular minimum and separates the 
scalar field up into distinct massive and massless Goldstone bosons.	
The non-zero expectation value of the scalar field leads mass terms to arise from mixtures of the gauge fields.
These mixtures are the real electroweak gauge bosons, 
\begin{align}
\begin{pmatrix}W^{+}_{\mu}\\W^{-}_{\mu}\end{pmatrix} = \frac{1}{\sqrt{2}}\begin{pmatrix}1&i\\1&-i\end{pmatrix}\begin{pmatrix}W^{1}_{\mu}\\W^{2}_{\mu}\end{pmatrix}  \, , \quad 
\begin{pmatrix}Z^{0}_{\mu}\\A_{\mu}\end{pmatrix} = \frac{1}{\sqrt{2}}\begin{pmatrix}\ctw&-\stw\\\stw&\ctw\end{pmatrix}\begin{pmatrix}W^{3}_{\mu}\\B_{\mu}\end{pmatrix}   \, ,
\end{align}
where the angle $\theta_W$ is the Weinberg angle and parametrises the mixing between the neutral bosons.
The masses of the \Wpm and the \Z  along with the mass term for the Higgs field can be written as 
\begin{align}
m_{W} = \frac{g_W \mu }{2\sqrt{\lambda}} , \quad m_{\Z} = \frac{m_W}{\sqrt{2}\ctw} \quad \text{and} \quad m_H = \sqrt{2}\mu \, .
\end{align}
The \Wpm and \Z bosons were first discovered at \cern by the UA1 experiment~\cite{Arnison:1983rp,Arnison:1983mk} and a new particle compatible with the Higgs boson 
with a mass of around $125\gev$ was observed at the \lhc~\cite{CMS:2012gu,ATLAS:2012gk}.

%There are several coupling terms between each of the electroweak bosons and the fermions of the form
%\begin{align}
%\bar{f} \gamma^\mu \Wpm_\mu f  \quad , \bar{f} \gamma^\mu \Z_\mu f
%\end{align}
%where $f$ is either a left-handed \grpsutw doublet or a right-handed singlet.
%Due to the spontaneously broken Higgs field, these couplings do not define the coupling
%between the electroweak gauge bosons and the mass eigenstates of each of the fermions. 
%The are the Yukawa couplings~\cite{Yukawa:1935xg} and are described in Section~\ref{sec:yukawa}.

\subsection{The flavour sector}

%The flavour sector describes the quarks and combinations thereof along with the strong force which interacts solely between the quarks.
The Lagrangian for the flavour sector is given by
\begin{align}
\mathcal{L} = \mathcal{L}_{Quarks} + \mathcal{L}_{Gluons}  + \mathcal{L}_{Yukawa} \, ,
\end{align}
where the first term describes the kinetics of the quarks and the second  
describes the gauge bosons for the strong force along with their interactions with the quarks.
The last term describes the interactions of the quarks and gluons with the Higgs field via Yukawa couplings which determine the 
mass eigenstates of the quarks.

The structure of the flavour sector is given by the \grpsuthree gauge group and each of the quarks form colour triplets, 
\begin{align}
q_C = \begin{pmatrix}q_R\\q_G\\q_B\end{pmatrix} \, ,
\end{align}
where there are three colour indices, $R$, $G$ and $B$.
Applying local gauge invariance to the \grpsuthree group 
requires a gauge field ($G_\mu^{a}$) of the strong force. 
The covariant derivative is constructed as
\begin{align}
D_\mu = \partial_\mu \mathbf{I} + i g_S \mathbf{T}^a G^{a}_\mu \, ,
\end{align}
where $g_S$ is the strong coupling constant and $\mathbf{T}^{a=1-8}$ are the 
\grpsuthree generators which obey the commutation relation,
\begin{align}
\mathbf{T}^a = i f^{a,b,c} \left[ \mathbf{T}^b, \mathbf{T}^c  \right] \, ,
\end{align}
where $f^{a,b,c}$ are the structure constants of \grpsuthree.
%	and hence the group is non-Abelian through the non-zero commutation relation.
The dynamics of the gluon field are given by the tensors
\begin{align}
G^a_{\mu\nu} &= \partial_\mu G^a_{\nu}  - \partial_\nu G^a_{\mu}  + g_S f^{a,b,c} G^b_{\mu}  G^c_{\nu}   \, ,
\end{align}
The massless part of the Lagrangian for the flavour sector is given by
\begin{align}
\mathcal{L} =  \bar{q}^C \left( i \gamma^\mu D_\mu \right) q^C +  - \frac{1}{4} G^a_{\mu\nu} G^{a,\mu\nu} \, ,
\end{align}
where the last term shows a self-coupling between the gluons.
%The non-Abelian nature of \grpsuthree allows for not only couplings between the gluons and the quarks,
% but also self-coupling between the gluons described by terms of $\order(G^3)$ and $\order(G^4)$.

\subsection{ Fermion masses: $\mathcal{L}_{Yukawa}$	}
\label{sec:yukawa}
The fermion masses can be introduced through the Higgs mechanism whilst retaining local gauge invariance.
The Lagrangian for the coupling of the fermion and Higgs fields is given by 
\begin{align}
\mathcal{L} =   m_\uquark^i \bar{\uquark}^i_R \uquark^i_L + m_\dquark^i \bar{\dquark}^i_R \dquark^i_L + m_e^i \bar{\ell}^i_R \ell^i_L
\end{align}
where $u,d$ are the `up' and `down' type quarks, $\ell$ represents the leptons and 
the index $i$ runs over the three generations of fermions.
The mass term $m_{\uquark,\dquark,\ell}^i = \frac{v}{\sqrt{2}} Y_{\uquark,\dquark,\ell}^i$ is a combination of 
the vacuum expectation value for the Higgs field and a unique Yukawa coupling~\cite{Yukawa:1935xg} to parametrise the mass.
In the SM there are no right-handed neutrinos, therefore the Lagrangian for neutrino mass eigenstates cannot be defined using this mechanism.

Each of the Yukawa terms, $Y^i_j$, can be written as a 3x3 dimensional matrix for the quarks and leptons.
In order to remove terms which couple between the fermion generation each of these Yukawa matrices must be diagonal.
The Yukawa matrices can by diagonalised by unitary matrices, $\mathcal{U}$, which acts on each fermion field through
\begin{align}
f_H = \mathcal{U}(f,H) f^{'}_H \, ,
\end{align}
for a given fermion $f$ of handedness $H$ leading to
\begin{subequations}\begin{align}
\mathcal{U}(\uquark,R)^\dagger Y_\uquark \mathcal{U}(\uquark,L) = \begin{pmatrix}m_u&0&0\\0&m_c&0\\0&0&m_t\end{pmatrix} \, , \\
\mathcal{U}(\dquark,R)^\dagger Y_\dquark \mathcal{U}(\dquark,L) = \begin{pmatrix}m_d&0&0\\0&m_s&0\\0&0&m_b\end{pmatrix} \, ,\\
\mathcal{U}(\ell,R)^\dagger Y_\ell \mathcal{U}(\ell,L) = \begin{pmatrix}m_e&0&0\\0&m_\mu&0\\0&0&m_\tau\end{pmatrix} \, , 
\end{align}\end{subequations}
where the mass eigenstates are given on the right side of the equations.

The left-handed fermion fields are \grpsutw doublets so these transformations act independently on
 each part of the doublet, i.e.
\begin{align}
\begin{pmatrix}u_L\\d_L\end{pmatrix} = 
\begin{pmatrix}\mathcal{U}(u,L) u_L^{'}\\ \mathcal{U}(d,L) d_L^{'}\end{pmatrix} = \,
\mathcal{U}(u,L) \begin{pmatrix}u_L^{'}\\V \, d_L^{'}\end{pmatrix}  \, ,
\end{align}
where the matrix $V = \mathcal{U}(u,L)^\dagger\mathcal{U}(d,L)$ describes the transformations between the left-handed up and down quarks.
The \grpsutw singlet representation of the right-handed fermions allows mass terms without mixing matrices.

The mixing matrix $V$ is the Cabbibo-Kobayashi-Maskawa (\ckm) matrix~\cite{Cabibbo:1963yz,Kobayashi:1973fv} which is a 3x3 
unitary matrix and completely specified by three mixing angles and a complex phase.
The \ckm matrix can be written both in terms of the 9 matrix elements $V_{ij}$ and also 
in terms of the Wolfenstein parametrisation~\cite{PDG2012},
\begin{align}
V_{\ckm} = 
\begin{pmatrix}
V_{ud} & V_{us} & V_{ub} \\
V_{cd} & V_{cs} & V_{cb} \\
V_{td} & V_{ts} & V_{tb} 
\end{pmatrix}
=
\begin{pmatrix}
1 - \lambda^2 / 2 & \lambda & A \lambda^3(\rho-i\eta) \\
 - \lambda &  1 - \lambda^2 / 2 & A\lambda^2 \\
A \lambda^2(1-\rho-i\eta)  & -A\lambda^2 & 1 
\end{pmatrix}
\end{align}
where the parameters $\rho, \eta, \lambda$ and $A$ are chosen such that the matrix is unitary of $\order(\lambda^4)$.

The unitary transformation allows the electroweak couplings with the quarks and the \Wpm bosons
but does not affect the coupling between the neutral bosons,
\begin{align}
\bar{\uquark}\, U_u^{\dagger}\, U_d\, \dquark\, \Wpm \, , \quad \bar{\dquark}\, U_d^{\dagger}\, U_d\, \dquark\, \Z \, ,
\end{align} 
where the charged current allows transitions between mass eigenstates for up and down type quarks 
but, due to the unitary nature of the mixing matrix, the neutral current does not mix with the mass eigenstates.
This gives flavour-changing charged currents at tree level but not flavour-changing neutral currents.
In order to construct flavour changing neutral currents, higher order diagrams with a loop containing a charged current
are required.
Flavour violating effects from particles beyond the SM can come from particles in these loop processes.

%\subsection{ New physics in the flavour sector }
%
%Contributions from physics beyond the SM, so called 'new physics', 
%can be parametrised by considering the SM as an effective theory. 
%An effective theory is one which works up to a given mass or energy scale, 
%of which Classical mechanics and the Fermi theory of electroweak interactions are two low energy examples.
%The current limits for the mass scale of the SM indicate that the bound is already $10^5\tev$ [ REFS ].
%However, the limit from W scattering and radiative corrections to the Higgs mass [ REFS ] 
% indicates that some new physics should appear at the \tev scale.

