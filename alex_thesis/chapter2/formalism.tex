\chapter{The formalism of particle physics}
\label{chap:formalism}

The Standard Model (SM) of particle physics~\cite{Glashow:1961tr,Weinberg:1967tq,Salam:1968rm,Higgs:1966ev,Guralnik:1964eu,Kibble:1967sv,Yukawa:1935xg,Cabibbo:1963yz,Kobayashi:1973fv}
 describes the behaviour and the interactions of each of the elementary particles known at the time of writing.
In the SM there are twelve matter particles of which there are six quarks and six leptons.
The fermions interact via three different forces, the electromagnetic, the weak and the strong force. 
The particles which mediate the fundamental forces (bosons) are the photon, the \Wpm and the \Z , and the gluon respectively~\cite{PDG2012}.
The mass of each particle, both fermions and bosons, is determined through the interaction with the Higgs field.

This chapter describes the basics on which the Standard Model is based. % and demonstrates the basics on which this thesis is based.
The flavour sector is described in detail along with a description of possible methods to incorporate physical effects beyond the scope of the Standard Model.
The \bquark\to\squark  penguin decay is introduced as a model-independent  test for 
contributions from new physical effects and the current experimental status of \BdToKstmm is presented.


\input{chapter2/standardmodel}

\input{chapter2/fcncs}


\section{Experimental results}

The first measurement of a \bquark\to\squark FCNC was the \btosgam transition observed in the measurement of the branching fraction of \decay{\B}{\Kstar\g} at \cleo in 1993~\cite{PhysRevLett.71.674}.
\BdKstGam is a radiative electroweak penguin decay described by the photon operator \Ope7 and hence is sensitive to \C7 .
Subsequent precision measurements of \BdKstGam and 
the similar decay \BsPhiGam have been performed by the B factories, \babar~\cite{babar:exp-b2kstgamma:2009} 
and \belle~\cite{PhysRevLett.103.241801} along with \lhcb~\cite{LHCb-PAPER-2011-042,LHCb-PAPER-2012-019,LHCb-CONF-2012-004}.
These measurements of the differential branching fraction of \BdKstGam, \BsPhiGam and measurements of 
the \CP asymmetry \ACP(\BdKstGam)~\cite{PDG2012} agree well with the predictions from the SM~\cite{hfag:2012,Ali:th-b2vgamma-NNLO:2008}.

The FCNC decay \BdToKstll was proposed as a further test for contribution from physics beyond the SM in Ref~\cite{PhysRevD.61.114028}.
However, the differential branching fraction of the inclusive decay \BdToKstll and the exclusive decay \BdToKstmm 
have been measured~\cite{hfag:2012} to be
\begin{align}
\Gamma(\BdToKstll) &= 9.9^{+1.2}_{-1.1} \times 10^{-7} \\
\Gamma(\BdToKstmm) &= 1.06\pm0.10 \times 10^{-6}
\end{align}
and are compatible with SM prediction~\cite{Ligeti:2007sn,Huber:2007vv}.

Further measurements of \BdToKstll are based on  evaluating the
angular distribution of the daughter particles to understand the \Kstarz polarisation amplitudes.
How to determine the maximal amount of information from the decay
while keeping uncertainties from QCD minimal has recently attracted much 
interest~\cite{Kruger:2005ep,Egede:2008uy,AltmannshoferBall,Egede:2010zc,Bobeth:2010wg,Matias:2012xw}.

The results from the experimental analyses of 
\BdToKstll~\cite{Aubert:2008ju,PhysRevLett.103.171801,Aaltonen:2011cn,Aaij:2011aa} 
have focused on the forward-backward asymmetry of the
dimuon system (\AFB) and the fraction of longitudinal polarisation of
the \Kstarz (\FL) as a function of the dimuon invariant mass.
The latest measurements from \babar, \belle and \cdf for \FL and \AFB are shown in Fig.~\ref{fig:otherexp}.
\begin{figure}[tbp]
\centering
\subfigure[\FL]{\includegraphics[width=0.48\columnwidth]{chapter2/figs/plot_FLExp.pdf}}
\subfigure[\AFB]{\includegraphics[width=0.48\columnwidth]{chapter2/figs/plot_AFBExp.pdf}}
\caption[\BdToKstmm measurements from \babar, \belle and \cdf]
{The fraction of longitudinal polarisation of the \Kstarz (\FL) and 
the  forward-backward asymmetry of the dimuon system (\AFB)  as measured by
\cdf~\cite{Aaltonen:2011cn,Aaltonen:2011ja}, \belle~\cite{PhysRevLett.103.171801} 
and \babar~\cite{Aubert:2007hz,Aubert:2008ju} 
along with the theoretical prediction from Ref.~\cite{Bobeth:2011gi}.~\label{fig:otherexp} }
\end{figure}
It is possible to see that there is some tension between these measurements of both \FL and \AFB at low dimuon invariant masses.
Contributions from physics beyond the SM have been predicted to change the \qsq spectrum of  \AFB~\cite{Egede:2010zc} which are not excluded by these measurements.
New measurements of \FL and \AFB are  needed to understand the exact shape of \AFB and clarify the discrepancy in the regions of low and high \qsq.
%In this thesis, the world-best measurements of the angular observables of \BdToKstmm using data from the \lhcb experiment presented along with 
%new measurements of the \kpi system.% do help understanding of the result of current and future angular analyses.

